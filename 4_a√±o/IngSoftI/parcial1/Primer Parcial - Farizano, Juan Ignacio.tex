\documentclass[11pt]{article}
\usepackage[a4paper, margin=0.5cm]{geometry}
\usepackage[utf8]{inputenc}
\usepackage[spanish, mexico]{babel}
\usepackage[spanish]{layout}
\usepackage[article]{ragged2e}
\usepackage{czt}


\title{
    Primer Parcial \\
    \large Ingeniería de Software I}
\author{Farizano, Juan Ignacio. Legajo: F-3562/9}

\date{}

\begin{document}

\maketitle

\rule{\textwidth}{1pt}

\section*{Designaciones}
\begin{itemize}
\item correo es el correo electrónico de un cliente $\approx$ CORREOELECTRONICO(correo)
\item nproducto es el nombre de un producto $\approx$ NOMBREPRODUCTO(nproducto)
\item cant representa una cantidad pedida o en stock de unidades un producto $\approx$ CANTIDAD(cant)
\item El conjunto de las posibles respuestas de una operación, haya fallado o no $\approx$ REPS
\item Aumenta en n el stock del producto p en el comercio $\approx$ AgregarStockAComercio(p, n)
\item Decrementa en n el stock del producto p en el comercio $\approx$ RestarStockAComercio(p, n)
\item Agrega al carrito n unidades del producto p en el carrito del cliente c y 
      decrementa en n el stock del producto p $\approx$ AgregarProducto(c, p, n)
\item Elimina el producto p del carrito del cliente c y restaura el stock antes pedido al comercio
$\approx$ EliminarProducto(c, p)
\item Reemplaza por n la cantidad de unidades pedidas del producto p en el carrito
del cliente c y modifica el stock del comercio según corresponda $\approx$ ModificarProducto(c, p, n)
\item Se crea un carrito vacío para que el cliente c inicie sus compras $\approx$ IniciarCarrito(c)
\item El cliente c finaliza su compra, se devuelve la orden y se borra el carrito $\approx$ FinalizarCompra(c)
\end{itemize}

\section*{Especificaciones}

\textbf{Nota:} \emph{type-checking} realizado utilizando el plug-in de Community Z Tools (CZT)
en el IDE Eclipse.

%% =====================================================================================
%% Tipos y esquemas de estado
%% =====================================================================================

\begin{zed}
  [NOMBREPRODUCTO, CORREOELECTRONICO] \\
  CANTIDAD == \nat \\
  REPS ::= ok | productoNoEncontrado | stockInsuficiente | carritoNoEncontrado | carritoYaIniciado | \\
  cantidadNoPositiva | cantidadNegativa | productoYaAgregado | productoNoAgregado | compraVacia
\end{zed}

\begin{schema}{Carrito}
  pedidos : NOMBREPRODUCTO \pfun CANTIDAD
\end{schema}
  
\begin{schema}{CarritoInicial}
Carrito
\where
pedidos = \emptyset
\end{schema}

\begin{schema}{Comercio}
stocks : NOMBREPRODUCTO \pfun CANTIDAD \\
carritos : CORREOELECTRONICO \pfun Carrito
\end{schema}

%% =====================================================================================
%% Marco de promoción
%% =====================================================================================


\begin{schema}{CarritoAComercio}
\Delta Comercio \\
\Delta Carrito \\
correo? : CORREOELECTRONICO
\where
correo? \in (\dom carritos) \\
carritos \; correo? = \theta Carrito \\
carritos' = carritos \oplus \{ correo? \mapsto \theta Carrito\ ' \} \\
stocks' = stocks
\end{schema}

\begin{schema}{ECarritoNoEncontrado}
\Xi Comercio \\
correo? : CORREOELECTRONICO \\
rep! : REPS
\where
correo? \notin (\dom carritos) \\
rep! = carritoNoEncontrado
\end{schema}

%% =====================================================================================
%% odificar stock del comercio
%% =====================================================================================

\begin{schema}{AgregarStockAComercioOk}
\Delta Comercio \\
nproducto? : NOMBREPRODUCTO \\
cant? : CANTIDAD
\where
cant? > 0 \\
nproducto? \in (\dom stocks) \\
stocks' = stocks \oplus \{ nproducto? \mapsto (stocks \; nproducto?) + cant? \} \\
carritos' = carritos
\end{schema}

\begin{schema}{EProductoNoEncontrado}
\Xi Comercio \\
nproducto? : NOMBREPRODUCTO \\
rep! : REPS
\where
nproducto? \notin (\dom stocks) \\
rep! = productoNoEncontrado
\end{schema}

%% No reviso datos internos de un esquema, se puede usar en comercio o carrito
\begin{schema}{ECantidadNoPositiva}
cant? : CANTIDAD \\
rep! : REPS
\where
cant? \leq 0 \\
rep! = cantidadNoPositiva
\end{schema}

\begin{zed}
AgregarStockAComercioE == EProductoNoEncontrado \lor ECantidadNoPositiva \\
AgregarStockAComercio == AgregarStockAComercioOk \lor AgregarStockAComercioE
\end{zed}

\begin{schema}{RestarStockAComercioOk}
\Delta Comercio \\
nproducto? : NOMBREPRODUCTO \\
cant? : CANTIDAD
\where
nproducto? \in (\dom stocks) \\
cant? \leq (stocks \; nproducto?) \\
stocks' = stocks \oplus \{ nproducto? \mapsto (stocks \; nproducto?) - cant? \} \\
carritos' = carritos
\end{schema}

\begin{schema}{EStockInsuficiente}
\Xi Comercio \\
nproducto? : NOMBREPRODUCTO \\
cant? : CANTIDAD
\where
nproducto? \in (\dom stocks) \\
cant? > (stocks \; nproducto?) \\
\end{schema}

\begin{zed}
RestarStockAComercioE == EProductoNoEncontrado \lor EStockInsuficiente \\
RestarStockAComercio == RestarStockAComercioOk \lor RestarStockAComercioE
\end{zed}

%% =====================================================================================
%% Agregar producto
%% =====================================================================================

\begin{schema}{CaAgregarProductoOk}
\Delta Carrito \\
nproducto? : NOMBREPRODUCTO \\
cant? : CANTIDAD \\
rep! : REPS
\where
nproducto? \notin (\dom pedidos) \\
cant? > 0 \\
pedidos' = pedidos \cup \{ nproducto? \mapsto cant? \} \\
rep! = ok
\end{schema}

\begin{schema}{EProductoYaAgregado}
\Xi Carrito \\
nproducto? : NOMBREPRODUCTO \\
rep! : REPS
\where
nproducto? \in (\dom pedidos) \\
rep! = productoYaAgregado
\end{schema}

\begin{zed}
CaAgregarProductoE == ECantidadNoPositiva \lor EProductoYaAgregado \\
AgregarProductoE == ECarritoNoEncontrado \lor EProductoNoEncontrado \lor CaAgregarProductoE \\
AgregarProductoOk == (CarritoAComercio \land CaAgregarProductoOk) \semi RestarStockAComercio\\
AgregarProducto == AgregarProductoOk \lor AgregarProductoE
\end{zed}

%% =====================================================================================
%% Eliminar producto
%% =====================================================================================

\begin{schema}{CaEliminarProductoOk}
\Delta Carrito \\
nproducto? : NOMBREPRODUCTO \\
cant! : CANTIDAD \\
rep! : REPS
\where
nproducto? \in (\dom pedidos) \\
cant! = pedidos \; nproducto? \\
pedidos' = \{ nproducto? \} \ndres pedidos \\
rep! = ok
\end{schema}

\begin{schema}{EProductoNoAgregado}
\Xi Carrito \\
nproducto? : NOMBREPRODUCTO \\
rep! : REPS
\where
nproducto? \notin (\dom pedidos) \\
rep! = productoNoAgregado
\end{schema}

\begin{zed}
CaEliminarProductoE == EProductoNoAgregado \\
EliminarProductoE == ECarritoNoEncontrado \lor EProductoNoEncontrado \lor CaEliminarProductoE \\
EliminarProductoOk == (CarritoAComercio \land CaEliminarProductoOk)[cant? / cant!] \semi AgregarStockAComercio \\
EliminarProducto == EliminarProductoOk \lor EliminarProductoE
\end{zed}

%% =====================================================================================
%% Modificar producto
%% =====================================================================================

\begin{schema}{CaModificarProductoMayorOk}
\Delta Carrito \\
nproducto? : NOMBREPRODUCTO \\
nuevaCant? : CANTIDAD \\
restarAStock! : CANTIDAD \\
rep! : REPS
\where
nproducto? \in (\dom pedidos) \\
nuevaCant? > 0 \\
nuevaCant? > (pedidos \; nproducto?) \\
pedidos' = pedidos \oplus \{ nproducto? \mapsto nuevaCant? \} \\
restarAStock! = nuevaCant? - (pedidos \; nproducto?) \\
rep! = ok
\end{schema}


\begin{schema}{CaModificarProductoMenorOk}
\Delta Carrito \\
nproducto? : NOMBREPRODUCTO \\
nuevaCant? : CANTIDAD \\
sumarAStock! : CANTIDAD \\
rep! : REPS
\where
nproducto? \in (\dom pedidos) \\
nuevaCant? > 0 \\
nuevaCant? < (pedidos \; nproducto?) \\
pedidos' = pedidos \oplus \{ nproducto? \mapsto nuevaCant? \} \\
sumarAStock! = (pedidos \; nproducto?) - nuevaCant? \\
rep! = ok
\end{schema}

\begin{schema}{CaModificarProductoCeroOk}
\Delta Carrito \\
nproducto? : NOMBREPRODUCTO \\
nuevaCant? : CANTIDAD \\
sumarAStock! : CANTIDAD \\
rep! : REPS
\where
nproducto? \in (\dom pedidos) \\
nuevaCant? = 0 \\
pedidos' = \{ nproducto? \} \ndres pedidos \\
sumarAStock! = pedidos \; nproducto?\\
rep! = ok
\end{schema}

\begin{schema}{ECantidadNegativa}
\Xi Carrito \\
nproducto? : NOMBREPRODUCTO \\
nuevaCant?: CANTIDAD \\
rep! : REPS
\where
nproducto? \in (\dom pedidos) \\
nuevaCant? < 0 \\
rep! = cantidadNegativa
\end{schema}


% Saltos de línea necesarios para que entre en la página del PDF
\begin{zed}
CaModificarProductoE == ECantidadNegativa \lor EProductoNoAgregado \\
ModificarProductoE == ECarritoNoEncontrado \lor EProductoNoEncontrado \lor CaModificarProductoE  \\
ModificarProductoMayor == (CarritoAComercio \land CaModificarProductoMayorOk)[cant? / restarAStock!] \semi \\
\; RestarStockAComercio \\
ModificarProductoMenorOCero == (CarritoAComercio \land (CaModificarProductoMenorOk \lor \\
\; CaModificarProductoCeroOk))[cant? / sumarAStock!] \semi AgregarStockAComercio \\
ModificarProductoOk == ModificarProductoMayor \lor ModificarProductoMenorOCero \\
ModificarProducto == ModificarProductoOk \lor ModificarProductoE
\end{zed}

%% =====================================================================================
%% Iniciar carrito
%% =====================================================================================

\begin{schema}{IniciarCarritoOk}
\Delta Comercio \\
correo? : CORREOELECTRONICO \\
rep! : REPS \\
CarritoInicial
\where
correo? \notin (\dom carritos) \\
carritos' = carritos \cup \{ correo? \mapsto \theta Carrito \} \\
stocks' = stocks \\
rep! = ok
\end{schema}

\begin{schema}{ECarritoYaIniciado}
\Xi Comercio \\
correo? : CORREOELECTRONICO \\
rep! : REPS \\
CarritoInicial
\where
correo? \in (\dom carritos) \\
rep! = carritoYaIniciado
\end{schema}

\begin{zed}
IniciarCarritoE == ECarritoYaIniciado \\
IniciarCarrito == IniciarCarritoOk \lor IniciarCarritoE
\end{zed}

%% =====================================================================================
%% Finalizar compra
%% =====================================================================================

\begin{schema}{FinalizarCompraOk}
\Delta Comercio \\
correo? : CORREOELECTRONICO \\
orden! : NOMBREPRODUCTO \pfun CANTIDAD \\
rep! : REPS
\where
correo? \in (\dom carritos) \\
(carritos \; correo?).pedidos \neq \emptyset \\
orden! = (carritos \; correo?).pedidos \\
carritos' = \{ correo? \} \ndres carritos \\
stocks' = stocks \\
rep! = ok
\end{schema}

\begin{schema}{ECompraVacia}
\Xi Comercio \\
correo? : CORREOELECTRONICO \\
rep! : REPS
\where
correo? \notin (\dom carritos) \\
(carritos \; correo?).pedidos = \emptyset \\
rep! = compraVacia
\end{schema}

\begin{zed}
FinalizarCompraE == ECarritoNoEncontrado \lor ECompraVacia \\
FinalizarCompra == FinalizarCompraOk \lor FinalizarCompraE
\end{zed}

\end{document}