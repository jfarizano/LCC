\documentclass[11pt]{article}
\usepackage[a4paper, margin=2.54cm]{geometry}
\usepackage[utf8]{inputenc}
\usepackage[spanish, mexico]{babel}
\usepackage[spanish]{layout}
\usepackage[article]{ragged2e}
\usepackage{czt}

\begin{document}
  \begin{enumerate}
    \item 
    Defino el tipo MENSAJES.
    \begin{zed}
      MENSAJES ::= ok | numeroClienteEnUso | clienteInexistente | montoNulo \\
                      | montoIncorrecto | noPoseeSaldoSuficiente | superaLimite
    \end{zed}
    \item Archivo aparte
    \item Placeholder
    \item
    \begin{schema}{DepositarOk}
      \Delta Banco \\
      d? : DNI \\
      m? : DINERO \\
      rep! : MENSAJES

      \where
      d? \in \dom ca \\
      200 \leq m? \\
      ca' = ca \oplus \{d? \mapsto (ca \; d?) + m?\} \\
      rep! = ok
    \end{schema}

    \begin{schema}{MontoIncorrecto}
      \Xi Banco \\
      m? : DINERO \\
      rep! : MENSAJES
      \where
      m? < 200 \\
      rep! = montoNulo
    \end{schema}
    \item Placeholder
    \item 
    \begin{axdef}
      limiteExtrCA : \nat
      \where
      limiteExtrCA = 1000
    \end{axdef}

    \begin{schema}{ExtraerCAOk}
      \Delta Banco \\
      d? : DNI \\
      m? : DINERO \\
      rep! : MENSAJES 
      \where
      d? \in \dom ca \\
      0 < m?\\
      m? \leq min \{ca \; d?, \; limiteExtrCA\}\\
      ca' = ca \oplus \{d? \mapsto (ca \; d?) - m?\} \\
      rep! = ok
    \end{schema}
    
    \begin{schema}{ClienteInexistente}
      \Xi Banco \\
      d? : DNI \\
      rep! : MENSAJES
      \where
      d? \notin \dom ca \\
      rep! = clienteInexistente
    \end{schema}

    \begin{schema}{MontoIncorrectoCA}
      \Xi Banco \\
      m? : DINERO \\
      rep! : MENSAJES
      \where
      m? < 0 \\
      rep! = montoNulo
    \end{schema} 

    \begin{schema}{SaldoInsuficiente}
      \Xi Banco \\
      d? : DNI \\
      m?: DINERO \\
      rep! : MENSAJES
      \where
      m? > ca \; d? \\
      d? \in \dom ca \\
      rep! = noPoseeSaldoSuficiente
    \end{schema}

    \begin{schema}{SuperaLimite}
      \Xi Banco \\
      m? : DINERO \\
      rep! : MENSAJES
      \where
      m? > limiteExtrCA \\
      rep! = superaLimite
    \end{schema}


    \begin{zed}
      ExtraerCAE == ClienteInexistente  \lor MontoIncorrectoCA \lor SaldoInsuficiente
      \lor SuperaLimite \\
      ExtraerCA = ExtraerCAOk \lor ExtraerCAE 
    \end{zed}

    \item Debería volver a usarse el esquema Extraer
    \item Modificamos ExtraerCAOk y SaldoInsuficiente
    \begin{schema}{ExtraerCAOk}
      \Delta Banco \\
      d? : DNI \\
      m? : DINERO \\
      rep! : MENSAJES 
      \where
      d? \in \dom ca \\
      0 < m?\\
      m? \leq min \{0.5 * ca \; d?, \; limiteExtrCA\}\\
      ca' = ca \oplus \{d? \mapsto (ca \; d?) - m?\} \\
      rep! = ok
    \end{schema}

    \begin{schema}{SaldoInsuficiente}
      \Xi Banco \\
      d? : DNI \\
      m?: DINERO \\
      rep! : MENSAJES
      \where
      m? > 0.5 * ca \; d? \\
      d? \in \dom ca \\
      rep! = noPoseeSaldoSuficiente
    \end{schema}

    \item
    \begin{schema}{PedirSaldoOk}
      \Xi Banco \\
      d? : DNI \\
      saldo! : DINERO \\
      rep! : MENSAJES
      \where
      d? \in \dom ca \\
      saldo! = ca \; d? \\
      rep! = ok
    \end{schema}

    \begin{zed}
      PedirSaldo == PedirSaldoOk \lor ClienteInexistente
    \end{zed}
  \end{enumerate}
\end{document}