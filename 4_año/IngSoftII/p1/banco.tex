\documentclass[11pt]{article}
\usepackage[a4paper, margin=1cm]{geometry}
\usepackage[utf8]{inputenc}
\usepackage{babel}
\usepackage[spanish]{layout}
\usepackage[article]{ragged2e}
\usepackage{textcomp}
\usepackage{amsmath}
\usepackage{amssymb}
\usepackage{amsfonts}
\usepackage{proof}
\usepackage{enumerate}
\usepackage{graphicx}
\usepackage{multirow}
\usepackage{caption}
\usepackage{subcaption}
\usepackage{amsthm}
\usepackage{zed-csp}
\usepackage[pdftex,colorlinks,linkcolor=blue]{hyperref}
\usepackage[cm]{fullpage}

\newcommand{\modulo}{{\bf Module }}

\newenvironment{module}[1]{\hypertarget{mi:#1}{}\vspace{0.5cm}\noindent\begin{tabular}{|p{0.2\linewidth} p{0.7\linewidth}|} \hline \modulo &{\bf #1} \\}{\hline\end{tabular}\vspace{0.5cm}}

\newenvironment{moduleNODP}[1]{\hypertarget{mi:#1}{}\vspace{0.5cm}\noindent\begin{tabular}{|p{0.2\linewidth} p{0.7\linewidth}|} \hline \modulo &{\bf #1} \\}{\hline\end{tabular}\vspace{0.5cm}}

\newenvironment{gmodule}[2]{\hypertarget{mi:#1}{} \vspace{0.5cm}\noindent\begin{tabular}{|p{0.25\linewidth} p{0.65\linewidth}|} \hline {\bf Generic Module} & {\bf #1(#2)} \\}{\hline\end{tabular}\vspace{0.5cm}}

\newenvironment{gmoduleNODP}[2]{\hypertarget{mi:#1}{} \vspace{0.5cm}\noindent\begin{tabular}{|p{0.25\linewidth} p{0.65\linewidth}|} \hline & \multicolumn{1}{r|}{\hyperlink{mg:#1}{\fbox{MG}}} \\ {\bf Generic Module} & {\bf #1(#2)} \\}{\hline\end{tabular}\vspace{0.5cm}}

\newenvironment{hmodule}[2]{\hypertarget{mi:#1}{} \vspace{0.5cm}\noindent\begin{tabular}{|p{0.2\linewidth} p{0.7\linewidth}|} \hline {\bf Module} & {\bf #1 inherits from} \hyperlink{mi:#2}{#2} \\}{\hline\end{tabular}\vspace{0.5cm}}

\newenvironment{hmoduleNODP}[2]{\hypertarget{mi:#1}{} \vspace{0.5cm}\noindent\begin{tabular}{|p{0.2\linewidth} p{0.7\linewidth}|} \hline & \multicolumn{1}{r|}{\hyperlink{mg:#1}{\fbox{MG}}} \\ {\bf Module} & {\bf #1 inherits from} \hyperlink{mi:#2}{#2} \\}{\hline\end{tabular}\vspace{0.5cm}}

\newcommand{\cmodule}[3]{\hypertarget{mi:#1}{} \vspace{0.5cm}\noindent\begin{tabular}{|p{0.2\linewidth} p{0.7\linewidth}|} \hline {\bf Module} & {\bf #1 is} \hyperlink{mi:#2}{#2}(\hyperlink{mi:#3}{#3})\\ \hline\end{tabular}\vspace{0.5cm}}

\newcommand{\cmoduleNODP}[3]{\hypertarget{mi:#1}{} \vspace{0.5cm}\noindent\begin{tabular}{|p{0.2\linewidth} p{0.7\linewidth}|} \hline & \multicolumn{1}{r|}{\hyperlink{mg:#1}{\fbox{MG}}}  \\ {\bf Module} & {\bf #1 is} \hyperlink{mi:#2}{#2}(\hyperlink{mi:#3}{#3})\\ \hline\end{tabular}\vspace{0.5cm}}

\newenvironment{pattern}[1]{\vspace{0.5cm}\noindent\begin{tabular}{|p{0.2\textwidth} p{0.75\textwidth}|} \hline{\bf Pattern} & {\bf #1} \\}{\hline\end{tabular}\vspace{0.5cm}}

\newenvironment{ipattern}[1]{\vspace{0.5cm}\noindent\begin{tabular}{|p{0.2\textwidth} p{0.75\textwidth}|} \hline{\bf Pattern} & {\bf #1} \\}{\hline\hline\end{tabular}\vspace{0.5cm}}

\newenvironment{fpattern}[1]{\vspace{0.5cm}\noindent\begin{tabular}{|p{0.2\textwidth} p{0.75\textwidth}|} \hline\hline & {\bf #1 (cont.)} \\}{\hline\end{tabular}\vspace{0.5cm}}

\newenvironment{mpattern}[1]{\vspace{0.5cm}\noindent\begin{tabular}{|p{0.2\textwidth} p{0.75\textwidth}|} \hline\hline & {\bf #1 (cont.)} \\}{\hline\hline\end{tabular}\vspace{0.5cm}}

\newcommand{\eproc}{{\bf exportsproc}}

\newcommand{\priv}{{\bf private}}

\newcommand{\proc}[1]{& #1 \\}

\newcommand{\e}[1]{{\bf i} \hyperlink{mi:#1}{#1}}

\newcommand{\eb}[1]{{\bf i} #1}

\newcommand{\s}[1]{{\bf o} \hyperlink{mi:#1}{#1}}

\newcommand{\es}[1]{{\bf io} \hyperlink{mi:#1}{#1}}

\newcommand{\compr}{{\bf comprises}}

\newcommand{\subm}[1]{& \hyperlink{mi:#1}{#1} \\}

\newcommand{\imp}[1]{{\bf imports} & #1 \\}

\newcommand{\minh}[1]{{\bf inherits from} & #1 \\}

\newcommand{\super}[1]{{\bf is parent of} & #1 \\}

\newcommand{\comm}[1]{{\bf comments} & #1 \\}

\newcommand{\mdr}[1]{\hyperlink{mi:#1}{#1}}

\newcommand{\rmg}[1]{\hyperlink{mg:#1}{#1}}

\newcommand{\rdp}[1]{\hypertarget{dp:#1}{}\hyperlink{mi:#1}{#1}}

\newcommand*{\ra}{\rightarrow}
\newcommand*{\new}{\text{{\bf new} }}
\newcommand*{\ec}{\mathbin{\Box}}

\newcommand{\based}[1]{{\bf based on} & #1 \\}

\newcommand{\assigns}{{\bf where}}

\newcommand{\is}[2]{& #1 {\bf is} #2 \\}

\newcommand{\dueto}[1]{{\bf because} & #1 \\}

\newcommand{\proto}[1]{{\bf protocol} & #1 \\}

\newcommand{\inp}{{\bf inports}}

\newcommand{\outp}{{\bf outports}}

\newcommand{\ann}{{\bf announces}}

\newcommand{\coe}{{\bf callonevent}}

\newcommand{\call}[2]{& #1 {\bf calls} #2 \\}


\setlength{\parindent}{0pt}

\title{Práctica 1 - Banco}
\author{}
\date{}

\begin{document}
\maketitle

\section{Enunciado}

Un banco posee varios tipos de clientes. Están aquellos que sólo poseen una o más cajas de ahorros en pesos, los que poseen además una o más cuentas corrientes y los que tienen uno o más depósitos a plazo fijo. Todo cliente debe consignar sus datos personales. Además a todo cliente que posea una caja de ahorros o una cuenta corriente el banco le envía por correo electrónico el resumen de todas sus cuentas. \\

La diferencia entre una caja de ahorros y una cuenta corriente es que sobre esta última sus titulares pueden librar cheques y, acordándolo previamente con el banco, pueden tener saldo negativo. Las operaciones sobre cuentas son las típicas: apertura, depósito, extracción, solicitud de saldo, transferencia a otra cuenta, cierre. Un depósito a plazo fijo tiene un monto inicial, una fecha de depósito, otra de vencimiento y un monto a pagar por el banco a sus titulares el día del vencimiento.


\section{Items de cambio}

\begin{itemize}
    \item Caja de ahorro.
    \item Cuenta corriente.
    \item Plazo fijo.
    \item Datos guardados por clientes.
    \item El envío de resumen por correo electrónico.
    \item Manejo de cuentas por cliente.
    \item ...
\end{itemize}

\section{Especificación de interfaces}

\begin{module}{CuentaBancaria}
\imp{\mdr{Monto}}

\eproc
\proc{depositar(\e{Monto})}
\proc{extraer(\e{Monto})}
\proc{consultarSaldo() : Monto}

\end{module}

\begin{hmodule}{CajaAhorro}{CuentaBancaria}
\end{hmodule}

\begin{hmodule}{CuentaCorriente}{CuentaBancaria}
\eproc
\proc{librarCheque(\e{Monto})}
\end{hmodule}


\begin{module}{PlazoFijo}
\imp{\mdr{FechaHora}, \mdr{Monto}}
\eproc
\proc{setMontoInicial(\e{Monto})}
\proc{getMontoInicial() : Monto}
\proc{setFechaDeposito(\e{FechaHora})}
\proc{getFechaDeposito() : FechaHora}
\proc{setFechaVencimiento(\e{FechaHora})}
\proc{getFechaVencimiento() : FechaHora}
\proc{setMontoAPagar(\e{Monto})}
\proc{getMontoAPagar() : Monto}
\end{module}

\begin{module}{DatosCliente}
\imp{\mdr{DNI}}
\eproc
\proc{getNombre() : String}
\proc{setNombre(\eb{String})}
\proc{getDNI() : DNI}
\proc{setDNI(\e{DNI})}
\proc{getDomicilio() :String}
\proc{setDomicilio(\e{String})}
\proc{getCorreo() : String}
\proc{setCorreo(\eb{String})}
\end{module}

\begin{gmodule}{Lista}{X}
\imp{X}
\eproc
\proc{add(\eb{X})}
\proc{head():X}
\proc{next():X}
\proc{more():Bool}
\proc{delete()}
\end{gmodule}

\cmodule{ListaCA}{Lista}{CajaAhorro}

\cmodule{ListaCC}{Lista}{CuentaCorriente}

\cmodule{ListaPF}{Lista}{PlazoFijo}

\begin{module}{Cliente}
\imp{\mdr{DatosCliente}, \mdr{CuentaCorriente}, \mdr{CajaAhorro}, \mdr{PlazoFijo}, \mdr{ListaCA}, \mdr{ListaCC}, \mdr{ListaPF}}
\eproc
\proc{Cliente(\e{DatosCliente})}
\proc{nuevaCC()}
\proc{nuevaCA()}
\proc{nuevoPF()}
\proc{getDatosCliente() : DatosCliente}
\proc{getCuentasCorrientes() : ListaCC}
\proc{getCajasAhorro() : ListaCA}
\proc{getPlazosFijos() : ListaPF}
\end{module}


% % \begin{gmodule}{Tuple}{X, Y}
% % \imp{X, Y}
% % \eproc
% % \proc{Tuple(\e{X}, \e{Y})}
% % \proc{first(): X}
% % \proc{second() : Y}
% % \end{gmodule}

% \cmodule{DatoResumen}{Tuple}{String, String}

% \cmodule{Resumen}{Lista}{DatoResumen}

\begin{module}{ResumenDeCuenta}
\eproc
\proc{getAtributo():String}
\proc{setAtributo(\eb String)}
\proc{getValor():String}
\proc{setValor(\eb String)}
\proc{firstAtributo():String}
\proc{nextAtributo()}
\proc{mas():Bool}
\proc{eliminar()}
\end{module}

\cmodule{ListaClientes}{Lista}{Cliente}

\begin{module}{GestionResumen}
\imp{\mdr{Cliente}, \mdr{CuentaBancaria}, \mdr{ResumenDeCuenta}}
\eproc
\proc{enviarResumen(\e{Cliente})}
\priv
\proc{obtenerResumen(\e{Cliente}) : ResumenDeCuenta}
\end{module}

\begin{module}{GestionClientes}
\imp{\mdr{Cliente}, \mdr{ListaClientes}, \mdr{DatosCliente}, \mdr{CuentaCorriente}, \mdr{CajaAhorro}, \mdr{PlazoFijo}}
\eproc
\proc{nuevoCliente(\e{DatosCliente})}

\end{module}

\begin{module}{MasterModule}
\imp{\mdr{GestionResumen}, \mdr{GestionClientes}}
\eproc
\proc{gestionar()}

\end{module}


\end{document}