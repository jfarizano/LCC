\documentclass[11pt]{article}
\usepackage[a4paper, margin=2cm]{geometry}
\usepackage[utf8]{inputenc}
\usepackage{babel}
\usepackage[spanish]{layout}
\usepackage[article]{ragged2e}
\usepackage{textcomp}
\usepackage{amsmath}
\usepackage{amssymb}
\usepackage{amsfonts}
\usepackage{proof}
\usepackage{enumerate}
\usepackage{graphicx}
\usepackage{multirow}
\usepackage{caption}
\usepackage{subcaption}

\setlength{\parindent}{0pt}

\title{
    Entrega 10 \\
    \large Sistemas Operativos II}
\author{Mellino, Natalia \and Farizano, Juan Ignacio}

\date{}

\begin{document}
\maketitle

\noindent\rule{\textwidth}{1pt}

\section*{Ejercicio 1}

\begin{itemize}

\item \textbf{Extensión:} \\
  ¿Cómo abrirlo? el nombre de cada archivo se divide en dos secciones separadas
  por un punto: el nombre en sí y su extensión. El sistema mantiene una lista de
  extensiones conocidas para las cuales sabe cómo actuar y se propaga este diseño
  a las aplicaciones que sólo abrirían aquellos archivos cuyas extensiones
  supieran manejar.
  
  ¿El tipo de archivo forma parte de los datos o metadatos? En este caso el tipo
  de archivo forma parte de los metadatos, ya que el nombre del archivo forma
  parte de la metadata que guarda el sistema sobre el archivo.
  
\item \textbf{Números mágicos:} \\
  ¿Cómo abrirlo? El sistema mantiene una lista compilada de las huellas digitales
  de los principales formatos que debe manejar, para así poder reconocer el tipo
  de un archivo basado en sus primeros bytes.
  
  ¿El tipo de archivo forma parte de los datos o metadatos? El tipo de archivo en 
  este caso forma parte de los datos, ya que estos números mágicos se encuentran
  dentro del archivo.
  
\item \textbf{Metadatos externos:} \\
  ¿Cómo abrirlo? Se separa el sistema de archivos en dos divisiones (forks):
  la data fork (los datos que propiamente constituyen a los archivos) y la resource fork
  (información acerca de los archivos), almacenada en una estructura independiente.
  
  ¿El tipo de archivo forma parte de los datos o metadatos? Forma parte de los
  metadatos, ya que se encuentran en la resource fork.

\end{itemize}

\section*{Ejercicio 2}

Los sistemas basados en Windows también soportan una estructura de directorios
basada en grafos y si bien los dos tipos de liga son válidos, es más común emplear los
accesos directos en vez de los mecanismos de liga. Estos accesos directos se
identifican por la extensión .lnk y fue creado principalmente para poder apuntar
a los archivos desde el escritorio y los menúes; si un proceso solicita al
sistema abrir el acceso directo, no obtendrá al archivo destino, sino al
acceso directo del mismo.

\section*{Ejercicio 3}

\begin{itemize}
\item \textbf{Oculto:} En Unix no se presenta como atributo en sí, si no que
      por convención un archivo o una carpeta es oculta y no es mostrada al usuario
      en el listado de directorio si su nombre empieza con un punto.
\item \textbf{Sólo lectura:} No se encuentra como un solo atributo, si no como una
      combinación de varios atributos. Un archivo es sólo lectura para un usuario si solo
      tiene el permiso r ('read') y no los permisos w ('write') y x ('execute')
      que se encuentran como atributos de archivo en Unix.
\item \textbf{Sistema:} No presenta equivalente en Unix.
\item \textbf{Archivado:} No presenta equivalente en Unix.
\end{itemize}

\section*{Ejercicio 4}

Sabemos que el primer sector del archivo se encuentra en el sector 4 del disco,
de esta forma el bit 4 del mapa de bits estará en 1. Como no tenemos más información
sobre cómo se encuentra el disco y la ubicación de los restantes 2 sectores (es decir,
si el archivo fue fragmentado o los sectores ocupados son contiguos), no
podemos asegurar qué otros bits estarán encendidos.

\section*{Ejercicio 5}

Como la FAT es una lista enlazada de clusters, podemos ver cuál archivo empieza 
en este cluster o ver cuál otro cluster apunta al 10, e ir rastreando para atrás
sucesivamente hasta encontrar el primer cluster al cual un archivo apunte. De 
esta forma podemos obtener el tamaño del archivo en bytes (que se encuentra como
un metadato en el catálogo raíz), luego contar
la cantidad clusters usados y multiplicarla por el tamaño del cluster (que es
fijo e igual para todos). Si es menor al tamaño del archivo entonces
el 5 es el valor válido y el archivo continúa, en caso contrario el archivo entero
ya fue leído y es válido el -1. 

\end{document}