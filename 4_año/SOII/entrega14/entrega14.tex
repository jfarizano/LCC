\documentclass[11pt]{article}
\usepackage[a4paper, margin=2cm]{geometry}
\usepackage[utf8]{inputenc}
\usepackage{babel}
\usepackage[spanish]{layout}
\usepackage[article]{ragged2e}
\usepackage{textcomp}
\usepackage{enumerate}


\setlength{\parindent}{0pt}

\title{
    Entrega 14 \\
    \large Sistemas Operativos II}
\author{Mellino, Natalia \and Farizano, Juan Ignacio}

\date{}

\begin{document}
\maketitle

\noindent\rule{\textwidth}{1pt}

\section*{Ejercicio 1}

Sabemos que un proceso cuando quiere ingresar a una sección crítica construye un paquete
que incluye, entre otras cosas, un timestamp con la hora actual. Entonces, cuando tenemos
dos procesos que quieren acceder simultaneamente a la misma región crítica se compara el 
timestamp de ambos procesos y gana el más bajo: si a uno de los procesos le llega el 
mensaje del otro y su timestamp es más bajo, éste responde Ok (lo cual habilitará al
otro proceso eventualmente). En caso contrario, la petición se encola, por lo que este
proceso entrará antes a la sección crítica.

\section*{Ejercicio 2}

El núcleo de Nachos sigue una estructura monolítica, esto se puede ver ya que
los diferentes subsistemas no corren en espacio de usuario. Para migrarlo a otra
estructura, por ejemplo microkernel, a grandes rasgos lo que se debería hacer es
mover los subsistemas que se puedan mover al espacio de usuario y extender la interfaz
del sistema para que estos puedan funcionar.

\section*{Ejercicio 3}

La relación principal entre los sistemas microkernel y los distribuidos es que
los primeros se prestan para el armado de los sistemas distribuidos. 

Por otro lado, ambos se asemejan en que están conformados por 'piezas' (en el caso de los sistemas 
microkernel) o 'nodos' (en el caso de los sistemas distribuidos), los cuales pueden ser reiniciados o
reemplazados en caso de que alguno falle o se caiga sin afectar al resto. Además, los dos sistemas 
resultan fáciles de extender: en el caso de los micronúcleos basta con agregar un proceso de usuario, y 
en los sistemas distribuidos basta con agregar un nodo más. 

\end{document}