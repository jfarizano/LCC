\documentclass[11pt]{article}
\usepackage[a4paper, margin=1.5cm]{geometry}
\usepackage[utf8]{inputenc}
\usepackage{babel}
\usepackage[spanish]{layout}
\usepackage[article]{ragged2e}
\usepackage{textcomp}
\usepackage{amsmath}
\usepackage{amssymb}
\usepackage{amsfonts}
\usepackage{proof}
\usepackage{enumerate}
\usepackage{graphicx}
\usepackage{multirow}

\setlength{\parindent}{0pt}

\title{
    Entrega 4 \\
    \large Sistemas Operativos II}
\author{Mellino, Natalia \and Farizano, Juan Ignacio}
\date{}

\begin{document}
\maketitle

\noindent\rule{\textwidth}{1pt}

\section*{Ejercicio 1}

\begin{itemize}
    \item \textbf{Ronda egoísta (SRR):} favorece a los procesos que ya han pasado
          tiempo ejecutando antes que a los recién llegados. Los nuevos
          procesos se forman en la cola de \emph{procesos nuevos} y se avanza
          únicamente con la cola de \emph{procesos aceptados}. Para SRR se
          emplean los parámetros $a$ y $b$ donde $a$ indica el ritmo al cual
          se incrementa la prioridad de los procesos en la cola de \emph{procesos nuevos}
          y $b$, el ritmo de incremento de prioridad para los \emph{procesos aceptados}.
          Cuando la prioridad de un proceso nuevo alcanza la prioridad de un proceso
          aceptado, el nuevo se vuelve aceptado. Si la cola de aceptados queda vacía, se
          acepta el proceso nuevo con mayor prioridad.
    \item \textbf{Multicolas con prioridad:} tenemos una cantidad arbitraria de colas que
          tienen distintas prioridades. En cada cola tenemos a su vez procesos con prioridades.
          El planificador elegirá únicamente entre los procesos que estén formados en la cola de 
          mayor prioridad. Sólo cuando estos procesos terminen (o sean enviados a alguna otra cola),
          el planificador continuará con aquellos que estén en las siguientes colas.
\end{itemize}

Ambas políticas serán iguales en el caso que tengamos procesos cortos ya que en la política
de multicolas con dos clases de prioridad estos procesos terminarán sus tareas sin haber 
sido degradados a la segunda cola de prioridad. Entonces, si pensamos esto en términos 
de SRR, la cola de mayor prioridad en la política de multicolas sería la cola de 
\emph{procesos aceptados} en SRR y la cola de menor prioridad sería la de \emph{procesos nuevos}. 
Por ello, al ser cortos los procesos, estos se ejecutarán por completo en la cola de mayor prioridad 
haciendo que esta actúe como la cola de \emph{procesos aceptados}.

\section*{Ejercicio 2}

perdon tuve que inventar cosas \\
% no me sale sta verga queda muy largo manejalo xd
% \begin{center}
%     \begin{tabular}{|c|c|c|}
%         \hline
%         & Aging & Retroalimentación Inversa \\
%         \hline
%         Ventajas & Procesos cortos pueden terminar \newline su ejecución sin ser degradados & xd \\
%         \hline
%         Desventajas & xd & xd \\
%         \hline
%     \end{tabular}
% \end{center}

\textbf{Aging:}

\begin{itemize}
    \item \textbf{Ventajas:} resulta favorecedor para los procesos cortos ya que
          pueden terminar sus tareas sin ser degradados.
    \item \textbf{Desventajas:} los procesos más largos son castigados e incluso podría llegar
          a sufrir inanición.
\end{itemize}

\textbf{Retroalimentación inversa:}

\begin{itemize}
    \item \textbf{Ventajas:} el promover un proceso a una cola de más alta prioridad
          permite evitar que dicho proceso sufra inanición.
    \item \textbf{Desventajas:} el promover varios procesos podría llegar a causar
          que la cola de prioridad más alta tenga más cantidad de procesos en ella
          haciendo que el tiempo que tarde cada uno sea más alto. NO SEEEEEE NO HAY
          UNA PORONGA EN EL LIBRO.
\end{itemize}

\section*{Ejercicio 3}

\end{document}