\documentclass[11pt]{article}
\usepackage[a4paper, margin=2.54cm]{geometry}
\usepackage[utf8]{inputenc}
\usepackage[spanish, mexico]{babel}
\usepackage[spanish]{layout}
\usepackage[article]{ragged2e}
\usepackage{czt}
\pagenumbering{gobble}


\title{
    Segundo Parcial \\
    \large Ingeniería de Software I}
\author{Farizano, Juan Ignacio. Legajo: F-3562/9}

\date{}

\begin{document}

\maketitle
\rule{\textwidth}{1pt}

\section*{Designaciones}
\begin{itemize}

\item El cuadro de texto llamado NOMBRE es llenado $\approx$ SeLlenaNombre
\item El cuadro de texto llamado NOMBRE es vaciado $\approx$ SeVacíaNombre
\item Se debe mostrar en el área de FOTO la foto correspondiente al fondo elegido $\approx$ MostrarFoto
\item Se debe mostrar en el área de FOTO una imagen en blanco $\approx$ FotoEnBlanco
\item Se debe reproducir el sonido asociado a la imagen f con volumen v y tono grave $\approx$ ReproducirSGrave(v, f)
\item Se debe reproducir el sonido asociado a la imagen f con volumen v y tono medio$\approx$ ReproducirSMedio(v, f)
\item Se debe reproducir el sonido asociado a la imagen f con volumen v y tono bajo $\approx$ ReproducirSBajo(v, f)
\item Se presiona el botón de DETALLES $\approx$ PresionaDetalles
\item Se presiona el botón de SONIDO $\approx$ PresionaSonido
\item Se selecciona la opción ROJO en COLORES $\approx$ PresionaCRojo
\item Se selecciona la opción AZUL en COLORES $\approx$ PresionaCAzul
\item Se selecciona la opción VERDE en COLORES $\approx$ PresionaCVerde
\item Se selecciona la opción GRAVE en MODO $\approx$ PresionaMGrave
\item Se selecciona la opción MEDIO en MODO $\approx$ PresionaMMedio
\item Se selecciona la opción BAJO en MODO $\approx$ PresionaMBajo
\item En la barra deslizable de VOLUMEN se selecciona la opción v $\approx$ SeleccionaVolumen(v)
\item En la lista desplegable de FONDOS se selecciona la opción f (donde 0 es Sin imagen) $\approx$ SeleccionaFondo(f)
\item Valor máximo de VOLUMEN $\approx$ Max
\item Se activa DETALLES $\approx$ ActivarDetalles
\item Se desactiva DETALLES $\approx$ DesactivarDetalles
\item Si está activo COLORES se desactiva y viceversa $\approx$ AlternarColores
\item Se desactiva COLORES $\approx$ DesactivarColores
\item Se activa MODO $\approx$ ActivarModo
\item Se desactiva MODO $\approx$ DesactivarModo
\item Se activa SONIDO $\approx$ ActivarSonido
\item Se desactiva SONIDO $\approx$ DesactivarSonido
\item Se debe reproducir un sonido $\approx$ ReproducirSonido
\end{itemize}
\section*{Tabla de control y visibilidad}

\begin{table}[h!]
\begin{center}
\begin{tabular}{|c|c|c|}
    \hline
    Evento & Control & Visibilidad \\ \hline
    SeLlenaNombre       & EC & S \\ \hline
    SeVacíaNombre       & EC & S \\ \hline
    MostrarFoto         & MC & S \\ \hline
    FotoEnBlanco        & MC & S \\ \hline
    ReproducirSGrave    & MC & S \\ \hline
    ReproducirSMedio    & MC & S \\ \hline
    ReproducirSBajo     & MC & S \\ \hline
    PresionaDetalles    & EC & S \\ \hline
    PresionaSonido      & EC & S \\ \hline
    PresionaCRojo       & EC & S \\ \hline
    PresionaCAzul       & EC & S \\ \hline
    PresionaCVerde      & EC & S \\ \hline
    PresionaMGrave      & EC & S \\ \hline
    PresionaMMedio      & EC & S \\ \hline
    PresionaMBajo       & EC & S \\ \hline
    SeleccionaVolumen   & EC & S \\ \hline
    SeleccionaFondo     & EC & S \\ \hline
    Max                 & EC & S \\ \hline
    ActivarDetalles     & MC & S \\ \hline
    DesactivarDetalles  & MC & S \\ \hline
    AlternarColores     & MC & S \\ \hline
    DesactivarColores   & MC & S \\ \hline
    ActivarModo         & MC & S \\ \hline
    DesactivarModo      & MC & S \\ \hline
    ActivarSonido       & MC & S \\ \hline
    DesactivarSonido    & MC & S \\ \hline
    ReproducirSonido    & MC & U \\ \hline
\end{tabular}
\end{center}
\end{table}

\section*{Especificaciones}


\end{document}