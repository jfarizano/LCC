\documentclass[11pt]{article}
\usepackage[a4paper, margin=1.5cm]{geometry}
\usepackage[utf8]{inputenc}
\usepackage[spanish, mexico]{babel}
\usepackage[spanish]{layout}
\usepackage[article]{ragged2e}
\usepackage{czt}

\begin{document}

\section*{Ejercicio 1}

\begin{zed}
	ESTADOLAMPARA ::= prendida | apagada
\end{zed}

\begin{schema}{Sistema}
	lampara : ESTADOLAMPARA
\end{schema}

\begin{schema}{SistemaInicial}
Sistema
\where
lampara = apagada
\end{schema}

\begin{schema}{AlternarEstadoBoton1}
\Delta Sistema
\where
lampara' \neq lampara
\end{schema}

\begin{zed}
AlternarEstadoBoton2 == AlternarEstadoBoton1
\end{zed}

\section*{Ejercicio 2}

\begin{zed}
	[MENSAJES] \\
	REPS ::= ok | paridadIncorrecta | bufferLleno | bufferVaciandose \\
	POSIBLESESTADOS ::= llenandose | vaciandose
\end{zed}

\begin{schema}{Despachante}
buffer : \seq{MENSAJES} \\
estado : POSIBLESESTADOS
\end{schema}

\begin{schema}{DespachanteInicial}
Despachante
\where
buffer = \emptyset \\
estado = llenandose
\end{schema}

\begin{axdef} 
CapacidadMaximaBuffer : \nat
\where
CapacidadMaximaBuffer = 10
\end{axdef}

\begin{axdef}
ParidadOk : \power MENSAJES
\where
ParidadOk \neq \emptyset
\end{axdef}

\begin{schema}{RecibirMensajeOk}
\Delta Despachante \\
m? : MENSAJES \\
r! : REPS
\where
estado = llenandose \\
\# buffer < CapacidadMaximaBuffer \\
m? \in ParidadOk \\
buffer' = buffer \cat \langle m? \rangle \\
estado' = estado \\
r! = ok
\end{schema}

\begin{schema}{EBufferVaciandose}
\Xi Despachante \\
r! : REPS
\where
estado = vaciandose \\
r! = bufferVaciandose
\end{schema}

\begin{schema}{EBufferLleno}
\Delta Despachante \\
r! : REPS 
\where
\# buffer = CapacidadMaximaBuffer \\
buffer' = buffer \\
estado' = vaciandose \\
r! = bufferLleno
\end{schema}

\begin{schema}{EParidadIncorrecta}
\Xi Despachante \\
m? : MENSAJES \\
r! : REPS
\where
m? \notin ParidadOk \\
r! = paridadIncorrecta
\end{schema}

\begin{zed}
RecibirMensajeError == EBufferVaciandose \lor EBufferLleno \lor EParidadIncorrecta \\
RecibirMensaje == RecibirMensajeOk \lor RecibirMensajeError
\end{zed}

\begin{schema}{VaciandoBuffer}
\Delta Despachante \\
r! : MENSAJES
\where
\# buffer > 0 \\
estado = vaciandose \\
buffer' = tail~buffer \\
r! = head~buffer \\
estado' = estado 
\end{schema}

\begin{schema}{NoVaciarBuffer}
\Xi Despachante \\
\where
estado = llenandose
\end{schema}

\begin{schema}{BufferVaciado}
\Delta Despachante \\
\where
estado = vaciandose \\
\# buffer = 0 \\
buffer' = buffer \\
estado' = llenandose
\end{schema}

\begin{zed}
VaciarBuffer == VaciandoBuffer \lor NoVaciarBuffer \lor BufferVaciado
\end{zed}

\section*{Ejercicio 3}

\begin{zed}
  [PELICULA, GUIONISTA, DIRECTOR] \\
  REPS ::= ok | noEncontrado | peliculaYaExiste
 \end{zed}
 
 \begin{schema}{BDCine}
 dir : DIRECTOR \rel PELICULA \\
 gui : GUIONISTA \rel PELICULA
 \end{schema}
 
 \begin{schema}{BDCineInicial}
 BDCine
 \where
 dir = \emptyset \\
 gui = \emptyset
 \end{schema}
 
 \begin{schema}{DirigidasPor}
 \Xi BDCine \\
 d? : DIRECTOR \\
 p! : \power PELICULA
 \where
 p! = \ran (\{d?\} \dres dir)
 \end{schema}
 
 \begin{schema}{GuionadasPor}
 \Xi BDCine \\
 g? : GUIONISTA \\
 p! : \power PELICULA
 \where
 p! = \ran (\{g?\} \dres gui)
 \end{schema}
 
 \begin{schema}{ModificarDirectorOk}
 \Delta BDCine \\
 d? : DIRECTOR \\
 nd? : DIRECTOR \\
 p? : PELICULA \\
 r! : REPS 
 \where
 (d?, p?) \in dir \\
 dir' = (dir \setminus \{(d?, p?)\}) \cup \{(nd?, p?)\} \\
 gui' = gui \\
 r! = ok
 \end{schema}
 
 \begin{schema}{NoEncontrado}
 \Xi BDCine \\
 d? : DIRECTOR \\
 p? : PELICULA \\
 r!: REPS
 \where
 (d?, p?) \notin dir \\
 r! = noEncontrado
 \end{schema}
 
 \begin{zed}
 ModificarDirector == ModificarDirectorOk \lor NoEncontrado
 \end{zed}
 
 \begin{schema}{AltaPeliOk}
 \Delta BDCine \\
 d? : \power DIRECTOR \\
 g? : \power GUIONISTA \\
 p? : PELICULA \\
 r! : REPS
 \where
 p? \notin (\ran dir) \\
 p? \notin (\ran gui) \\
 dir' = dir \cup (d? \cross \{p?\}) \\
 gui' = gui \cup (g? \cross \{p?\}) \\
 r! = ok
 \end{schema}
 
 \begin{schema}{PeliYaExiste}
 \Xi BDCine \\
 p? : PELICULA \\
 r! : REPS
 \where
 (p? \in (\ran dir)) \lor (p? \in (\ran gui)) \\
 r! = peliculaYaExiste
 \end{schema}
 
 \begin{zed}
 AltaPeli == AltaPeliOk \lor PeliYaExiste
 \end{zed}
 
 \begin{schema}{BajaPeliOk}
 \Delta BDCine \\
 p? : PELICULA \\
 r! : REPS
 \where
 p? \in (\ran dir) \\
 p? \in (\ran gui) \\
 dir' = dir \nrres \{ p? \} \\
 gui' = gui \nrres \{ p? \} \\
 r! = ok
 \end{schema}
 
 \begin{schema}{PeliNoExiste}
 \Xi BDCine \\
 p? : PELICULA \\
 r! : REPS
 \where
 (p? \notin (\ran dir)) \lor (p? \notin (\ran gui)) \\
 r! = noEncontrado
 \end{schema}
 
 \begin{zed}
 BajaPeli == BajaPeliOk \lor PeliNoExiste
 \end{zed}

\section*{Ejercicio 4}

\begin{zed}
  [USUARIO, CONTRASENA, NOMBREARCHIVO, ARCHIVO] \\
  REPS ::= ok | usuarioNoEncontrado | usuarioYaConectado | contrasenaIncorrecta | usuarioNoConectado \\
        | archivoNoEncontrado | archivoYaExiste
  \end{zed}
  
  \begin{schema}{ServidorFTP}
  logins : USUARIO \pfun CONTRASENA \\ 
  usuariosConectados : \power USUARIO \\
  archivosSubidos : NOMBREARCHIVO \pfun ARCHIVO
  \end{schema}
  
  \begin{schema}{connectOk}
  \Delta ServidorFTP \\
  usuario? : USUARIO \\
  contrasena? : CONTRASENA \\
  r! : REPS
  \where
  usuario? \in (\dom logins) \\
  usuario? \notin usuariosConectados \\
  contrasena? = logins \; usuario? \\
  usuariosConectados' = usuariosConectados \cup \{ usuario? \} \\
  r! = ok
  \end{schema}
  
  \begin{schema}{EUsuarioNoEncontrado}
  \Xi ServidorFTP \\
  usuario? : USUARIO \\
  r! : REPS
  \where
  usuario? \notin (\dom logins) \\
  r! = usuarioNoEncontrado
  \end{schema}
  
  \begin{schema}{EUsuarioYaConectado}
  \Xi ServidorFTP \\
  usuario? : USUARIO \\
  r! : REPS
  \where
  usuario? \in (\dom logins) \\
  usuario? \in usuariosConectados \\
  r! = usuarioNoEncontrado
  \end{schema}
  
  \begin{schema}{EContrasenaIncorrecta}
  \Xi ServidorFTP \\
  usuario? : USUARIO \\
  contrasena? : CONTRASENA \\
  r! : REPS
  \where
  usuario? \in (\dom logins) \\
  contrasena? \neq logins \; usuario? \\
  r! = usuarioNoEncontrado
  \end{schema}
  
  \begin{zed}
  connectE == EUsuarioNoEncontrado \lor EUsuarioYaConectado \lor EContrasenaIncorrecta \\
  connect == connectOk \lor connectE
  \end{zed}
  
  \begin{schema}{getOk}
  \Xi ServidorFTP \\
  usuario? : USUARIO  \\
  narchivo? : NOMBREARCHIVO \\
  archivo! : ARCHIVO \\
  r! : REPS
  \where
  usuario? \in usuariosConectados \\
  narchivo? \in (\dom archivosSubidos) \\
  archivo! = archivosSubidos \; narchivo? \\
  r! = ok
  \end{schema}
  
  \begin{schema}{EUsuarioNoConectado}
  \Xi ServidorFTP \\
  usuario? : USUARIO \\
  r! : REPS
  \where
  usuario? \notin usuariosConectados \\
  r! = usuarioNoConectado
  \end{schema}
  
  \begin{schema}{EArchivoNoEncontrado}
  \Xi ServidorFTP \\
  narchivo? : NOMBREARCHIVO \\
  r! : REPS
  \where
  narchivo? \notin (\dom archivosSubidos) \\
  r! = archivoNoEncontrado
  \end{schema}
  
  \begin{zed}
  getE == EUsuarioNoConectado \lor EArchivoNoEncontrado \\
  get == getOk \lor getE
  \end{zed}
  
  \begin{schema}{putOk}
  \Delta ServidorFTP \\
  usuario? : USUARIO \\
  narchivo? : NOMBREARCHIVO \\
  archivo? : ARCHIVO \\
  r! : REPS
  \where
  usuario? \in usuariosConectados \\
  narchivo? \notin (\dom archivosSubidos) \\
  archivosSubidos' = archivosSubidos \cup \{ (narchivo?, archivo?) \} \\
  r! = ok
  \end{schema}
  
  \begin{schema}{EArchivoYaExiste}
  \Xi ServidorFTP \\
  narchivo? : NOMBREARCHIVO \\
  r! : REPS
  \where
  narchivo? \in (\dom archivosSubidos) \\
  r! = archivoYaExiste
  \end{schema}
  
  \begin{zed}
  putE == EUsuarioNoConectado \lor EArchivoYaExiste \\
  put == putOk \lor putE
  \end{zed}
  
  \begin{schema}{closeOk}
  \Delta ServidorFTP \\
  usuario? : USUARIO \\
  r! : REPS
  \where
  usuario? \in usuariosConectados \\
  usuariosConectados' = usuariosConectados \setminus \{ usuario? \} \\
  r! = ok
  \end{schema}
  
  \begin{zed}
  closeE == EUsuarioNoConectado \\
  close == closeOk \lor closeE
  \end{zed}


\section*{Ejercicio 6}

\begin{zed}
  [USUARIO, NOMBREPROGRAMA, ENCABEZADO, CUERPO] \\
  ESTADOEDIT ::= siendoEditado | disponible \\
  REPS ::= ok | programaNoExiste | operacionNoPermitida | programaSiendoEditado \\ 
  | programaNoSiendoEditado | userNoEstabaEditando
  \end{zed}
  
  \begin{schema}{Commit}
  encabezado: ENCABEZADO \\
  cuerpo: CUERPO
  \end{schema}
  
  \begin{schema}{Programa}
  historial : \seq Commit \\
  estado : ESTADOEDIT \\
  siendoEditadoPor : \power USUARIO
  \end{schema}
  
  \begin{schema}{ProgramaInicial}
  Programa
  \where
  historial = \emptyset \\
  estado = disponible \\
  siendoEditadoPor = \emptyset
  \end{schema}
  
  \begin{schema}{CVS}
  lineaBase : NOMBREPROGRAMA \pfun Programa \\
  lectores : \power USUARIO \\
  editores : \power USUARIO \\
  autores : \power USUARIO 
  \where
  autores \subseteq editores \\
  editores \subseteq lectores
  \end{schema}
  
  \begin{schema}{CVSInicial}
  CVS
  \where
  lineaBase = \emptyset \\
  lectores = \emptyset \\
  editores = \emptyset \\
  autores = \emptyset
  \end{schema}
  
  \begin{schema}{ProgramaACVS}
  \Delta CVS \\
  nprograma? : NOMBREPROGRAMA
  \where
  nprograma? \in (\dom lineaBase) \\
  lineaBase \; nprograma? = \theta Programa \\
  lineaBase' = lineaBase \oplus \{ nprograma? \mapsto \theta Programa' \} \\
  lectores' = lectores \\
  editores' = editores \\
  autores' = autores
  \end{schema}
  
  \begin{schema}{EProgramaNoExiste}
  \Xi CVS \\
  nprograma? : NOMBREPROGRAMA \\
  r! : REPS
  \where
  nprograma? \notin (\dom lineaBase) \\
  r! = programaNoExiste
  \end{schema}
  
  \begin{schema}{EUsuarioNoAutor}
  \Xi CVS \\
  user? : USUARIO \\
  r! : REPS
  \where
  user? \notin autores \\
  r! = operacionNoPermitida
  \end{schema}
  
  \begin{schema}{EUsuarioNoEditor}
  \Xi CVS \\
  user? : USUARIO \\
  r! : REPS
  \where
  user? \notin editores \\
  r! = operacionNoPermitida
  \end{schema}
  
  \begin{schema}{EUsuarioNoLector}
  \Xi CVS \\
  user? : USUARIO \\
  r! : REPS
  \where
  user? \notin lectores \\
  r! = operacionNoPermitida
  \end{schema}
  
  \begin{schema}{EProgSiendoEditado}
  \Xi CVS \\
  nprograma? : NOMBREPROGRAMA \\
  r! : REPS
  \where
  nprograma? \in (\dom lineaBase) \\
  (lineaBase \; nprograma?).estado = siendoEditado \\
  r! = programaSiendoEditado
  \end{schema}
  
  \begin{schema}{CreateOk}
  \Delta CVS \\
  nprograma? : NOMBREPROGRAMA \\
  user? : USUARIO \\
  r! : REPS
  \where
  (nprograma? \notin (\dom lineaBase)) \lor \\
  \; \; (nprograma? \in (\dom lineaBase) \land (lineaBase \; nprograma?).estado = disponible) \\
  user? \in autores \\
  lineaBase' = \\
  \; \; lineaBase \oplus \{ nprograma? \mapsto \lblot historial == \emptyset , estado == disponible , siendoEditadoPor == \emptyset \rblot \} \\
  r! = ok
  \end{schema}
  
  \begin{zed}
    Create == CreateOk \lor EProgSiendoEditado \lor EUsuarioNoAutor
  \end{zed}
  
  \begin{schema}{ProgGetOk}
  \Xi Programa \\
  prog! : Commit \\
  r! : REPS
  \where
  prog! = last~historial \\
  r! = ok
  \end{schema}
  
  \begin{zed}
  Get == ([ProgramaACVS ; user? : USUARIO | user? \in lectores] \land ProgGetOk) \\
  \lor (EUsuarioNoLector \lor EProgramaNoExiste)
  \end{zed}
  
  \begin{schema}{ProgEditOk}
  \Delta Programa \\
  user? : USUARIO \\
  prog! : Commit \\
  r! : REPS
  \where
  estado = disponible \\
  historial' = historial \\
  estado' = siendoEditado \\
  siendoEditadoPor' = \{ user? \} \\
  prog! = last~historial \\
  r! = ok
  \end{schema}
  
  \begin{zed}
  Edit == ([ProgramaACVS ; user? : USUARIO | user? \in editores] \land ProgEditOk) \lor \\
  (EUsuarioNoEditor \lor EProgramaNoExiste \lor EProgSiendoEditado)
  \end{zed}
  
  \begin{schema}{ProgDeltaOkDiff}
  \Delta Programa \\
  user? : USUARIO \\
  prog? : Commit \\
  r! : REPS 
  \where
  estado = siendoEditado \\
  siendoEditadoPor = \{ user? \} \\
  prog? \neq last~historial \\
  historial' = historial \cat \langle prog? \rangle \\
  estado' = disponible \\
  siendoEditadoPor' = \emptyset \\
  r! = ok
  \end{schema}
  
  \begin{schema}{ProgDeltaOkSame}
  \Delta Programa \\
  user? : USUARIO \\
  prog? : Commit \\
  r! : REPS 
  \where
  estado = siendoEditado \\
  siendoEditadoPor = \{ user? \} \\
  prog? = last~historial \\
  historial' = historial \\
  estado' = disponible \\
  siendoEditadoPor' = \emptyset \\
  r! = ok
  \end{schema}
  
  \begin{schema}{EProgNoSiendoEditado}
  \Xi Programa \\
  r! : REPS
  \where
  estado = disponible \\
  r! = programaNoSiendoEditado
  \end{schema}
  
  \begin{schema}{EUserNoEstabaEditando}
  \Xi Programa \\
  user? : USUARIO \\
  r! : REPS
  \where
  estado = siendoEditado \\
  siendoEditadoPor \neq \{ user? \} \\
  r! = userNoEstabaEditando
  \end{schema}
  
  \begin{zed}
  ProgDeltaOk == ProgDeltaOkDiff \lor ProgDeltaOkSame \\
  ProgDelta == ProgDeltaOk \lor EProgNoSiendoEditado \lor EUserNoEstabaEditando \\
  Delta == ([ProgramaACVS ; user? : USUARIO | user? \in editores] \land ProgDelta) \lor \\
  (EUsuarioNoEditor \lor EProgramaNoExiste)
  \end{zed}
  
  \begin{schema}{DeleteOk}
  \Delta CVS \\
  nprograma? : NOMBREPROGRAMA \\
  user? : USUARIO \\
  r! : REPS
  \where
  nprograma? \in (\dom lineaBase) \\
  user? \in autores \\
  lineaBase' =  \{ nprograma? \} \ndres lineaBase  \\
  lectores' = lectores \\
  editores' = editores \\
  autores' = autores \\
  r! = ok
  \end{schema}
  
  \begin{zed}
  Delete == DeleteOk \lor EUsuarioNoAutor \lor EProgramaNoExiste
  \end{zed}

\end{document}