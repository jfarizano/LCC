\documentclass[11pt]{article}
\usepackage[a4paper, margin=2.54cm]{geometry}
\usepackage[utf8]{inputenc}
\usepackage[spanish, mexico]{babel}
\usepackage[spanish]{layout}
\usepackage[article]{ragged2e}
\usepackage{textcomp}
\usepackage{amsmath}
\usepackage{amssymb}
\usepackage{amsfonts}
\usepackage{mathtools}

\DeclarePairedDelimiter\ceil{\lceil}{\rceil}
\DeclarePairedDelimiter\floor{\lfloor}{\rfloor}
\setlength{\parindent}{5pt}
\renewcommand{\baselinestretch}{1.5}

% ============================================================================
% ============================================================================
% ============================================================================

\title{Entrega Práctica 5 EDyA II}
\author{Farizano, Juan Ignacio}
\date{}

% ============================================================================
% ============================================================================
% ============================================================================

\begin{document}
\maketitle

\section*{Análisis de costos}
\subsection*{Función cons:}

Sea $h$ la altura del árbol, las recurrencias del trabajo y la profundidad
para la función cons son:
\begin{equation*}
  W_{cons}(h) = W_{cons}(h - 1) + 1 = (W_{cons}(h - 2) + 1) + 1 = ... =
  W_{cons}(1) + 1 + ... + 1 = \sum_{i=1}^{h} i
\end{equation*}
\begin{equation*}
  S_{cons}(h) = S_{cons}(h - 1) + 1
\end{equation*}

\noindent Supongo que $W_{cons}(h) \in O(h)$ y demuestro por el método de
substitución, luego: 
\begin{equation*}
  W_{cons}(h) = W_{cons}(h - 1) + 1 \underset{\text{HI}}{\leq} c \cdot (h-1) + 1
  = c \cdot h - c + 1 \underset{c \geq 1}{\leq} c \cdot h - c + c = c \cdot h
\end{equation*}
  
Por lo tanto $ W_{cons}(h) \in O(h) $ (análogo para $S_{cons}(h) \in O(h)$)

% ============================================================================

\subsection*{Función tabulate:}
Sea $n$ el tamaño del árbol y $f$ la función dada como argumento,
supongo que la complejidad de esta función es una constante $ k_f$.
La recurrencia del trabajo de la función tabulate es:
\begin{equation*}
  W_{tabulate}(n) = W_{tabulate}(\floor*{\frac{n}{2}}) + k_f + 
  W_{tabulate}(\floor*{\frac{n}{2}}) = 2 \cdot W_{tabulate}(\floor*{\frac{n}{2}})
  + k_f 
\end{equation*}

\noindent Supongo que $ W_{tabulate}(n) \in O(n) $ y demuestro por método
de substitución, luego: 
\begin{equation*}
  W_{tabulate}(n) = 2 \cdot W_{tabulate}(\floor*{\frac{n}{2}}) + k_f
  \underset{\text{HI}}{\leq} 2 \cdot c \cdot \floor*{\frac{n}{2}} + k_f \leq 
  2 \cdot c \cdot \frac{n}{2} + k_f \leq c \cdot n
\end{equation*}

Por lo tanto $ W_{tabulate}(n) \in O(n) $

La recurrencia de la profundidad de la función tabulate es:
\begin{equation*}
  S_{tabulate}(n) = max(S_{tabulate}(\floor*{\frac{n}{2}}),
  S_{tabulate}(\floor*{\frac{n}{2}})) + k_f  = S_{tabulate}(\floor*{\frac{n}{2}})
  + k_f
\end{equation*}

\noindent Supongo que $ S_{tabulate}(n) \in O(lg \; n) $ y demuestro por método
de substitución, luego:
\begin{align*}
  S_{tabulate}(n) &= S_{tabulate}(\floor*{\frac{n}{2}}) + k_f 
  \underset{\text{HI}}{\leq} c \; \cdot lg(\floor*{\frac{n}{2}}) + k_f
  \leq c \; \cdot lg(\frac{n}{2}) + k_f = c \; \cdot lg \; n +
  c \; \cdot lg \; 2 + k_f \\
  & = c \; \cdot lg \; n - c + k_f \leq c \cdot lg \; n
\end{align*}

Por lo tanto $ S_{tabulate}(n) \in O(lg \; n) $

% ============================================================================

\subsection*{Función take:}
Sea $h$ la altura del árbol.
\begin{equation*}
  W_{take}(h) = W_{take}(h - 1) + 1 = (W_{take}(h - 2) + 1) + 1 = ... =
  W_{take}(1) + 1 + ... + 1 = \sum_{i=1}^{h} i 
\end{equation*}
\begin{equation*}
  S_{take}(h) = S_{take}(h - 1) + 1
\end{equation*}

La demostración del costo de la función take es análoga a la de la función cons,
por lo tanto $ W_{take}(h) \in O(h) $ y $ S_{take}(h) \in O(h) $



\end{document}