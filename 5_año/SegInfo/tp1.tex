% - 1 ejercicio a elección del conjunto {3, 4}
% - 1 ejercicio a elección del conjunto {7, 8, 9}
% - 1 ejercicio a elección del conjunto {13, 14, 15, 18}
% - 1 inciso a elección del ejercicio 21, y el mismo inciso para el ejercicio 22
% - 1 desafío a elección del ejercicio 23

\documentclass[11pt]{article}
\usepackage[a4paper, margin=1.5cm]{geometry}
\usepackage[utf8]{inputenc}
\usepackage[spanish, mexico]{babel}
\usepackage[spanish]{layout}
\usepackage[article]{ragged2e}

\title{
    TP 1 \\
    \large Seguridad Informática}
\author{Farizano, Juan Ignacio}

\date{}

\begin{document}

\maketitle

\rule{\textwidth}{1pt}

\section*{Ejercicio 3}
Si las políticas de seguridad son muy restrictivas, con el e-mail por ejemplo, dificultaría
la comunicación con direcciones ajenas a las internas de la empresa, incluso provocando
que mails importantes no sean recibidos o terminen en la casilla de spam.

También si las politicas de acceso a los datos son muy firmes, podría dificultar
el desarrollo de nuevos productos, por ejemplo si un sector de una empresa está
desarrollando un nuevo producto que depende de una tecnología que está siendo 
a su vez desarrollada en un sector diferente, el intercambio de información limitado
provocaría que se retrase la fecha de salida de este producto.


\section*{Ejercicio 9}
Un ejemplo donde la ocultación de información no provee mayor seguridad es en el 
caso del software libre. Por ejemplo el kernel de Linux, sobre el cual corre la gran
mayoría de servidores del mundo, el hecho de que cualquiera pueda ver el código
permite que una gran desarrolladores de todo el mundo revise este mismo en busca
de vulnerabilidades y sean corregidas rápidamente.

Un ejemplo donde sí sucede es cuando se oculta la ubicación de información
confidencial. Por ejemplo, en una computadora se divide el disco en dos particiones,
una con información pública y otra con información sensible. Un usuario con un 
nivel de seguridad bajo no debería conocer la ubicación, tamaño o ni siquiera la
existencia de la partición con información de la que se necesita una nivel de seguridad
mayor para acceder a ella, esto dificulta el acceso no autorizado a ciertos datos.

\section*{Ejercicio 15}
Ver archivo \textbf{Ej15.hs} incluido

\section*{Ejercicios 21 y 22}
Me gusta el programa 4
\subsection*{Ejercicio 21 inciso ?}
\subsection*{Ejercicio 22 inciso ?}

\section*{Ejercicio 23}

\end{document}
