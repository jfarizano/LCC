\documentclass[11pt]{article}
\usepackage[a4paper, margin=2.54cm]{geometry}
\usepackage[utf8]{inputenc}
\usepackage[spanish, mexico]{babel}
\usepackage[spanish]{layout}
\usepackage{amsmath}
\usepackage{amssymb}
\usepackage{amsfonts}
\usepackage{tikz}
\usetikzlibrary{arrows}

\title{
    Entrega Práctica de Búsqueda \\
    \large Introducción a la Inteligencia Artificial}
\author{Juan Ignacio Farizano}
\date{}

\begin{document}
\maketitle
\noindent\rule{\textwidth}{1pt}

\section*{Ejercicio 9}
Formulo el problema en términos de espacios de estado:
\begin{itemize}
  \item \textbf{Conjunto de estados:} Represento el estado actual mediante un número
                                      $n \in \mathbb{Z}$
  \item \textbf{Estado inicial:} 1
  \item \textbf{Estado meta:} 7
  \item \textbf{Operadores:}
      Sea n el estado actual, si n no es divisible por 3 podemos realizar
      únicamente estas operaciones: \\
      \begin{center}
      \begin{tabular}{| c | c | c |}
      \hline
      \textbf{Operador} & \textbf{Regla} & \textbf{Costo} \\ \hline
      UNO & $ n \rightarrow 1 $ & 1 \\ \hline
      DOBLE & $ n \rightarrow n*2 $ & n \\ \hline
      SUMAR\_UNO & $ n \rightarrow n + 1 $ & 1 \\ \hline
      RESTAR\_UNO & $ n \rightarrow n - 1 $ & 1 \\ \hline
      \end{tabular}
      \end{center}
      Cuando n es divisible por 3 la única operación disponible es:
      \begin{center}
        \begin{tabular}{| c | c | c |}
        \hline
        \textbf{Operador} & \textbf{Regla} & \textbf{Costo} \\ \hline
        DIVIDIR & $ n \rightarrow \frac{n}{3} $ & $\frac{2n}{3}$ \\ \hline
        \end{tabular}
        \end{center}
\end{itemize}
Utilizando la heurística h(n) = $\| 7 - n \|$ busco una solución al problema
mediante el algoritmo $A^*$, dibujo el árbol de búsqueda:

\begin{center}
\begin{tikzpicture}[->,>=stealth',level/.style={sibling distance = 3.5cm/#1,
  level distance = 1.7cm}] 
\node [circle, draw] {1}
  child { node[circle, draw, red] {1}
  }
  child { node[circle, draw] {2}
          child { node[circle, draw, red] {1}   
          }
          child { node[circle, draw] {4} 
                  child { node[circle, draw, red] {1}
                  }
                  child { node[circle, draw] {8}
                          child {node[circle, draw, red] {1}
                          }
                          child {node[circle, draw] {16}
                          }
                          child {node[circle, draw] {9}
                          }
                          child {node[circle, draw, green] {7}
                          }
                  }
                  child[missing]
                  child[missing]
                %   child[missing]
                  child { node[circle, draw] {5}
                          child {node[circle, draw, red] {1}
                          }
                          child {node[circle, draw] {10}
                          }
                          child {node[circle, draw, red] {6}
                          }
                          child {node[circle, draw, red] {4}
                          }
                  }
                  child { node[circle, draw, red] {3}
                  }
          }
        %   child [missing]
          child [missing]
          child { node[circle, draw] {3}
                  child [missing]
                  child { node[circle, draw, red] {1}
                  }   
          }
          child { node[circle, draw, red] {1}
          }
  }
  %   child[missing]
  child { node[circle, draw, red] {2}
  }
  child { node[circle, draw] {0}
%     child [missing]
    child { node[circle, draw, red] {1}
    }
    child { node[circle, draw, red] {0}
    }
    child { node[circle, draw, red] {1}
    }
    child { node[circle, draw] {-1}
    }
  }
  
; 
\end{tikzpicture}
\end{center}

Para evitar estados repetidos, descarto los n que ya fueron encontrados, ya que
ya fueron recorridos con menor o igual costo anteriormente, estos son los nodos
coloreados en rojo. El estado meta es el nodo coloreado en verde.
En la siguiente tabla muestro los costos obtenidos en el recorrido, donde cada
estado fue agregado en el orden en que fue recorrido, y en el caso de tener
dos nodos con el mismo valor de f, recorro el que fue agregado primero.

\begin{center}
  \begin{tabular}{| c | c | c | c |}
  \hline
  \textbf{n} & \textbf{g(n)} & \textbf{h(n)} & \textbf{f(n)} \\ \hline
  1 & 0 & 6 & 6 \\ \hline
  2 & 1 & 5 & 6 \\ \hline
  0 & 1 & 7 & 8 \\ \hline
  4 & 3 & 3 & 6 \\ \hline
  3 & 2 & 4 & 6 \\ \hline
  8 & 7 & 1 & 8 \\ \hline
  5 & 4 & 3 & 7 \\ \hline
  10 & 9 & 3 & 12 \\ \hline
  -1 & 2 & 8 & 10 \\ \hline
  16 & 15 & 9 & 24 \\ \hline
  9 & 8 & 2 & 10 \\ \hline
  7 & 8 & 0 & 8 \\ \hline
  \end{tabular}
  \end{center}

\end{document}