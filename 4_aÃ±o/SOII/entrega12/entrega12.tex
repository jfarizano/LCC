\documentclass[11pt]{article}
\usepackage[a4paper, margin=2cm]{geometry}
\usepackage[utf8]{inputenc}
\usepackage{babel}
\usepackage[spanish]{layout}
\usepackage[article]{ragged2e}
\usepackage{textcomp}
\usepackage{amsmath}
\usepackage{amssymb}
\usepackage{amsfonts}
\usepackage{proof}
\usepackage{enumerate}
\usepackage{graphicx}
\usepackage{multirow}
\usepackage{caption}
\usepackage{subcaption}

\setlength{\parindent}{0pt}

\title{
    Entrega 12 \\
    \large Sistemas Operativos II}
\author{Mellino, Natalia \and Farizano, Juan Ignacio}

\date{}

\begin{document}
\maketitle

\noindent\rule{\textwidth}{1pt}

\section*{Ejercicio 1}
Una solución posible es simplemente borrar ambos archivos. Este sería
un método rápido y fácil de realizar, lleva la FAT a un estado consistente
pero se pierde información. \\
Otra posible solución sería hacer una copia de ambos archivos y borrar
los originales. No se pierde información pero probablemente uno de los archivos
esté corrupto. Además, esta solución es más lenta y ocupa más espacio en disco.

\section*{Ejercicio 2}
En UNIX no podrían ocurrir los \emph{cross-linked} files, ya que en este file system
no se encuentran cadenas de clusters, si no que simplemente los sectores utilizados
se marcan en el bitmap y luego en los \emph{inodos} se referencian todos los sectores
en vez de solo el primero. \\

Si la secuencia de pasos para crear un archivo es análoga al sistema FAT, entonces
al marcar primero los bloques utilizados en el bitmap y luego crear el inodo no es
posible que una entrada de directorio termine referenciando un bloque que ya está 
referenciado por otra entrada. En caso contrario, no podemos asegurar lo mismo.

\section*{Ejercicio 3}
Algunas ventajas destacables del journaling, es que a diferencia de los sistemas de 
archivos basados en logs, se necesita menos espacio para el journal y resulta ser más 
rápido en operaciones aleatorias (tanto lectura como escritura).

Por otro lado, una clara desventaja es que ante una falla abrupta no es posible
recuperar los datos de un archivo (sí los metadatos).

\end{document}