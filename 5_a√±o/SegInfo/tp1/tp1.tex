% - 1 ejercicio a elección del conjunto {3, 4}
% - 1 ejercicio a elección del conjunto {7, 8, 9}
% - 1 ejercicio a elección del conjunto {13, 14, 15, 18}
% - 1 inciso a elección del ejercicio 21, y el mismo inciso para el ejercicio 22
% - 1 desafío a elección del ejercicio 23

\documentclass[11pt]{article}
\usepackage[a4paper, margin=1.5cm]{geometry}
\usepackage[utf8]{inputenc}
\usepackage[spanish, mexico]{babel}
\usepackage[spanish]{layout}
\usepackage[article]{ragged2e}

\title{
    TP 1 \\
    \large Seguridad Informática}
\author{Farizano, Juan Ignacio}

\date{}

\begin{document}

\maketitle

\rule{\textwidth}{1pt}

\section*{Ejercicio 3}
Si las políticas de seguridad son muy restrictivas, con el e-mail por ejemplo, dificultaría
la comunicación con direcciones ajenas a las internas de la empresa, incluso provocando
que mails importantes no sean recibidos o terminen en la casilla de spam,
entorpeciendo o evitando que se den actividades comerciales.

También si las politicas de acceso a los datos son muy firmes, podría dificultar
el desarrollo de nuevos productos, por ejemplo si un sector de una empresa está
desarrollando un nuevo producto que depende de una tecnología que está siendo 
a su vez desarrollada en un sector diferente, el intercambio de información limitado
provocaría que se retrase la fecha de salida de este producto.


\section*{Ejercicio 9}
Un ejemplo donde la ocultación de información no provee mayor seguridad es en el 
caso del software libre. Por ejemplo el kernel de Linux, sobre el cual corre la gran
mayoría de servidores del mundo, el hecho de que cualquiera pueda ver el código
permite que una gran desarrolladores de todo el mundo revise este mismo en busca
de vulnerabilidades y sean corregidas rápidamente.

Un ejemplo donde sí sucede es cuando se oculta la ubicación de información
confidencial. Por ejemplo, en una computadora se divide el disco en dos particiones,
una con información pública y otra con información sensible. Un usuario con un 
nivel de seguridad bajo no debería conocer la ubicación, tamaño o ni siquiera la
existencia de la partición con información de la que se necesita una nivel de seguridad
mayor para acceder a ella, esto dificulta el acceso no autorizado a estos datos.

\section*{Ejercicio 15}
Ver archivo \textbf{Ej15.hs} incluido

\section*{Ejercicios 21 y 22}
Elijo el programa 4 para resolver los ejercicios 21 y 22.
\subsection*{Ejercicio 21}
No hay flujo indebido de información ya que en ningún momento se lee una variable
de nivel H, por lo que no puede haber ningún flujo indebido a algún canal de salida L.
A pesar de que se pueda obtener información sobre el usuario y la contraseña, estos
no están detrás de variables de nivel H, por lo que es más un problema de malas
prácticas de programación. 
\subsection*{Ejercicio 22}
Bajo el sistema de seguridad por multi-ejecución, este programa funcionaría siempre
de una sola manera por lo explicado en el inciso anterior. Al no haber ninguna lectura
o escritura de nivel H, el proceso original nunca realiza un fork, por lo tanto
la memoria nunca será diferente para usuarios con nivel L o con nivel H y la
ejecución será la misma para ambos. 

\section*{Ejercicio 23}
Mi desafío elegido es el step 7. 

En este desafío el lenguaje implementa un 
modelo simple de asignación dinámica de memoria, donde se le asigna un valor a una celda
de memoria. El problema con este sistema es que revisa que no se escriban valores
H en una referencia declarada como low, pero no revisa que no se escriba en una
variable L haciendo una deferencia a una variable de referencia high. 

Se puede explotar esta falla en el lenguaje, esta es mi solución para el problema:
\begin{verbatim}
    declare ref x : high;
    ref x = h;
    l = deref(x);
\end{verbatim}

Este es un ataque de filtrado de información con un flujo de información 
indirecto y explícito. Es indirecto porque el dato
pasó por una variable intermedia (la variable de referencia x) para ir de la
variable h a la variable l, y explícito porque el flujo de un dato hacia otro
fue de manera explícita en el código, realizando asignaciones de variables.

Para impedir el ataque se deberían modificar las reglas de inferencia del sistema
de tipos para que al hacer una asignación del estilo (x = deref(y)) se verifique
que el nivel de la variable x sea mayor o igual al nivel de la variable de referencia y.

\end{document}
