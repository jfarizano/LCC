\documentclass[11pt]{article}
\usepackage[a4paper, margin=2.54cm]{geometry}
\usepackage[utf8]{inputenc}
\usepackage[spanish, mexico]{babel}
\usepackage[spanish]{layout}
\usepackage[article]{ragged2e}
\usepackage{textcomp}
\usepackage{amsmath}
\usepackage{amsfonts}

\setlength{\parindent}{5pt}
\renewcommand{\baselinestretch}{1.5}

% ============================================================================
% ============================================================================
% ============================================================================

\title{Entrega Práctica 4 EDyA II}
\author{Farizano, Juan Ignacio}
\date{}

% ============================================================================
% ============================================================================
% ============================================================================

\begin{document}
\maketitle

\section*{Ejercicio 4:}

\noindent \textbf{tad} PriorityQ (a : Set) \textbf{where} \\
\indent \textbf{import} Bool \\
\indent vacia : PriorityQ a \\
\indent poner : a $\rightarrow$ N $\rightarrow$ PriorityQ a $\rightarrow$ PriorityQ a \\
\indent primero : PriorityQ a $\rightarrow$ a \\
\indent sacar : PriorityQ a $\rightarrow$ PriorityQ a \\
\indent esVacia : PriorityQ a $\rightarrow$ Bool \\
\indent union : PriorityQ a $\rightarrow$ PriorityQ a $\rightarrow$ PriorityQ a

% ============================================================================

\subsection*{Especificación algebraica}
\begin{verbatim}
poner x p (poner y p c) = poner y p c

primero (poner x p vacia) = x
primero (poner x p (poner y q c)) = if x > y then primero (poner x p c)
                                             else primero (poner y q c)

sacar (poner x p vacia) = vacia
sacar (poner x p (poner y q c)) = if p > q then poner y q (sacar (poner x p c))
                                           else poner x p (sacar (poner y q c))

esVacia vacia = True
esVacia (poner x p q) = False

union c vacia = c
union c (poner x p q) = union (poner x p c) q
\end{verbatim}

% ============================================================================

\subsection*{Especificación tomando como modelo los conjuntos}
\noindent vacia = \{\} \\
poner $x$ $p$ $\{(x_1,p_1),...,(x_n,p_n)\}$ = $\{(x_1,p_1),...,(x_n,p_n)\}$
si $p \in \{p_1,...,p_n\}$  \\
poner $x$ $p$ $\{(x_1,p_1),...,(x_n,p_n)\}$ = $\{(x_1,p_1),...,(x_n,p_n), (x, p)\}$
si $p \notin \{p_1,...,p_n\}$  \\
primero $\{(x_1,p_1),...,(x_n,p_n)\}$ = $(x_i,p_i)$ tal que $p_i = max(p_1,...,p_n)$ \\
sacar A = A - $\{(x_i, p_i) \in A / p_i = max(p_1,...,p_n)\}$ \\
esVacia $\{(x_1,p_1),...,(x_n,p_n)\} = True $ si $n = 0$ \\
esVacia $\{(x_1,p_1),...,(x_n,p_n)\} = False$ si $n > 0$ \\
union $\{(x_1,p_1),...,(x_n,p_n)\} \; \{(y_1,q_1),...,(y_m,q_m)\} =
\{(x_1,p_1),...,(x_n,p_n)\} \cup \{(y_i,q_i) / q_i \notin \{p_1,...,p_n\}\} $

% ============================================================================
% ============================================================================
% ============================================================================

\section*{Ejercicio 9:}
\noindent Dado el tipo de datos \textbf{data} AGTree a = Node a [AGTRee a],
defino su principio de inducción estructural: \\
Dada una propiedad $P$ sobre $AGTree$, para probar $\forall t::AGTree. \; P(t)$:
\begin{itemize}
  \item Probamos $P(Node \; a \; [])$
  \item Probamos que si $P(x_i) \; \forall i = 1,...,n$ entonces $P(Node \; a \; [x_1,...,x_n])$
\end{itemize}

% ============================================================================
% ============================================================================
% ============================================================================

\section*{Ejercicio 13:}

\noindent Definimos el siguiente tipo de datos \\
\textbf{type} Rank = Int \\
\textbf{data} Heap a = E $\vert$ N Rank a (Heap a) (Heap a) \\

\noindent El \textbf{rango} de un heap es la longitud de la espina derecha (el camino hacia
la derecha hasta un nodo vacío.) \\
Un leftist heap es una variante de heap cuya invariante es que el rango de
cualquier hijo izquierdo es mayor o igual que el de su hermano de la derecha.
Dado este tipo de datos, definimos las siguientes funciones:
\begin{verbatim}
merge :: Ord a => Heap a -> Heap a -> Heap a
merge h1 E = h1
merge E h2 = h2
merge h1@(N _ x a1 b1) h2@(N _ y a2 b2) = if x <= y then makeH x a1 (merge b1 h2)
                                                    else makeH y a2 (merge h1 b2)

rank :: Heap a -> Rank
rank E = 0
rank (N r _ _ _) = r

makeH x a b = if rank a >= rank b then N (rank b + 1) x a b
                                  else N (rank a + 1) x b a
\end{verbatim}

\noindent\textbf{Proposición:} Probar que si $l_1$ y $l_2$ son leftist heaps, entonces
merge $l_1$ $l_2$ es un leftist heap. \\
Para probar utilizaremos inducción estructural, primero definimos el principio de
inducción estructural de este tipo de datos: \\
Dada una propiedad $P$ sobre $Heap$, para probar $\forall \; h::Heap. \; P(h)$:
\begin{itemize}
  \item Probamos $P(E)$
  \item Probamos que si $P(h_1)$ y $P(h_2)$ entonces $P(N \; Rank \; a \; h_1 \; h_2)$
\end{itemize}
\textbf{Demostración por inducción estructural sobre $l_1$}: \\
Sea $l_1$ un leftist heap, luego $P(l_1):$ merge $l_1\;l_2$ es un leftist heap
$ \forall \; l_2$ leftist heap. 

\noindent $\bullet$ Caso base: $l_1\;=\;E$ \\
\indent \indent Sea $l_2$ un leftist heap. \\
\indent \indent merge $l_1 \; l_2 \; \underset{l_1 = \text{E}}{=} $ merge E $l_2 \underset{merge. 2}{=} l_2$ \\

\noindent $\bullet$ Caso inductivo: $l_1 = N \; r_1 \; x \; a_1 \; b_1$ \\
Sea $l_2$ un leftist heap. \\
\textbf{Hipótesis inductivas:}
\begin{itemize}
  \item H1: merge $a_1$ $l_2$ es un leftist heap $ \forall \; l_2$ leftist heap. 
  \item H2: merge $b_1$ $l_2$ es un leftist heap $ \forall \; l_2$ leftist heap. 
\end{itemize}
Utilizamos inducción estructural sobre $l_2$:

$\bullet$ Si $l_2 =$ E: \\
\indent \indent merge $l_1 \; l_2 \; \underset{l_2 = \text{E}}{=} $ merge $l_1$ E $\underset{merge. 1}{=} l_1$

$\bullet$ Si $l_2 = N \; r_2 \; y \; a_2 \; b_2$ tenemos dos posibilidades:
$x \leq y$ ó $x > y$: \\
Para el caso inductivo de $l_2$ tenemos nuevas \textbf{hipótesis inductivas:}
\begin{itemize}
  \item H3: merge $l_1$ $a_2$ es un leftist heap $ \forall \; l_1$ leftist heap. 
  \item H4: merge $l_1$ $b_2$ es un leftist heap $ \forall \; l_1$ leftist heap.
\end{itemize}

Vemos primero si $x \leq y$: \\
merge $l_1$ $l_2$ $\underset{merge. 3\;\text{y}\;x \leq y}{=}$ makeH $x \; a_1$
(merge $b_1 \; l_2$). Por H2 merge $b_1 \; l_2$ es un leftist heap, y por \textbf{Lema 1}
$\forall \; l_1 \; l_2$ leftist heap $\forall x$ . makeH $x \; l_1 \; l_2$
es un leftist heap, resultando ser makeH $x \; a_1$ (merge $b_1 \; l_2$) un
leftist heap. \\

Por último, si $x > y$: \\
merge $l_1$ $l_2$ $\underset{merge. 3\;\text{y}\;x > y}{=}$ makeH $y \; a_2$
(merge $l_1 \; b_2$). Por H4 merge $l_1 \; b_2$ es un leftist heap, y por \textbf{Lema 1}
$\forall \; l_1 \; l_2$ leftist heap $\forall x$ . makeH $x \; l_1 \; l_2$
es un leftist heap, resultando ser makeH $x \; a_2$ (merge $l_1 \; b_2$) un
leftist heap. \\

\begin{center}
  Por lo tanto, merge $l_1 \; l_2$ es un leftist heap.
\end{center}

Solo queda probar el lema auxiliar:

\textbf{Lema 1:} $\forall \; l_1 \; l_2$ leftist heap $\forall x$ . makeH $x \; l_1 \; l_2$
es un leftist heap.

\textbf{Demostración:} \\
Sean $l_1$ y $l_2$ leftist heaps, hay dos casos posbiles: rank $l_1 \geq$
rank $l_2$ ó rank $l_1 <$ rank $l_2$.

$\bullet$ Si rank $l_1 \geq$ rank $l_2$: \\
makeH $x \; l_1 \; l_2$ = $ N \; (rank \;l_2+1) \; x \; l_1 \; l_2$. Ya que rank
$l_1 \geq l_2$, se cumple la invariante y makeH $x \; l_1 \; l_2$ es un
leftist heap. 

$\bullet$ Si rank $l_1 <$ rank $l_2$: \\
makeH $x \; l_1 \; l_2$ = $ N \; (rank \;l_1+1) \; x \; l_2 \; l_1$. Ya que rank
$l_1 < l_2$, se cumple la invariante y makeH $x \; l_1 \; l_2$ es un
leftist heap.

\begin{center}
  Por lo tanto, el \textbf{Lema 1} queda demostrado.
\end{center}

\end{document}