\documentclass[11pt]{article}
\usepackage[a4paper, margin=2.54cm]{geometry}
\usepackage[utf8]{inputenc}
\usepackage[spanish, mexico]{babel}
\usepackage[spanish]{layout}
\usepackage[article]{ragged2e}
\usepackage{textcomp}
\usepackage{amsmath}
\usepackage{amsfonts}
\usepackage{yhmath}

\setlength{\parindent}{0pt}

\title{Trabajo Práctico: Unidad 5}
\author{Mellino, Natalia \and Farizano, Juan Ignacio}
\date{}



\begin{document}
\maketitle

%===============================================================================
\section*{Ejercicio 1:}

\textbf{\underline{Apartado a):}} \\

Sabemos que como $ f $ es una función de densidad de probabilidad se debe cumplir
la siguiente condición:

\begin{equation*}
  \int_{-\infty}^{+\infty} f(x) \cdot dx = 1
\end{equation*}

Es decir, debemos ver que:

\begin{align*}
  \int_{-2}^{2} k(2 - x)dx                                  & = 1           \\
  k \cdot \int_{-2}^{2} 2dx - k \cdot \int_{-2}^{2} x \cdot & = 1           \\
  8k - k \Big(\frac{(2)^2}{2} - \frac{(-2)^2}{2} \; \Big)   & = 1           \\
  8k                                                        & = 1           \\
  k                                                         & = \frac{1}{8}
\end{align*}

Por lo tanto, nuestra función nos queda de la siguiente manera:

\[
  f(x) = \begin{cases}
    \frac{1}{8}(2 - x) & \text{si }  -2 \leq x \leq 2 \\
    0                  & \text{en caso contrario }
  \end{cases}
\]

\textbf{\underline{Apartado b):}} \\

Función de distribución acumulada:

\begin{equation*}
  F(X) = P(X \leq x) = \int_{-2}^{x} \frac{1}{8}(2 - x)dx
\end{equation*}

\begin{align*}
  \int_{-2}^{x} \frac{1}{8}(2 - x)dx & = \frac{1}{4}(x + 2) - \frac{1}{8} \int_{-2}^{x} tdt            \\
                                     & = \frac{1}{4}(x + 2) - \frac{1}{8} \Big(\frac{x^2}{2} - 2 \Big) \\
                                     & = \frac{-x^2}{16} + \frac{1}{4}x + \frac{3}{4}
\end{align*}

Por lo tanto la Función de Distribución Acumulada es:

\[
  F(x) = \begin{cases}
    \frac{-x^2}{16} + \frac{1}{4}x + \frac{3}{4} & \text{si }  -2 \leq x \leq 2 \\
    0                  & \text{en caso contrario }
  \end{cases}
\]

\newpage

\textbf{\underline{Apartado c):}} \\

Queremos hallar $ P(-1 \leq X \leq 1) $, para ello planteamos:

\begin{align*}
  \int_{-1}^{1} f(x)dx & = \int_{-1}^{1} \frac{1}{8} (2-x) dx                                          \\
                       & = \frac{1}{4}(2) - \frac{1}{8} \Big( \frac{(1)^2}{2} - \frac{(-1)^2}{2} \Big) \\
                       & = \frac{1}{2}
\end{align*}

Por lo tanto, la probabilidad de que el termómetro cometa un error es de $\frac{1}{2}$. \\

\textbf{\underline{Apartado d):}} \\

\begin{align*}
  P(X > c) & = 1 - P(X \leq c)                                             \\
           & = 1 - F(c)                                                    \\
           & = 1 - \Big[\frac{-c^2}{16} + \frac{1}{4}c + \frac{3}{4} \Big]
\end{align*}

Luego queremos hallar $c$ tal que:

\begin{equation*}
  \Big[\frac{c^2}{16} - \frac{1}{4}c + \frac{1}{4} \Big] = 0.1
\end{equation*}

Esto ocurre sí y sólo sí:

\begin{equation*}
  c \simeq 3.2649 \; \; \text{ó} \; \; c \simeq 0.735
\end{equation*}

Como se debe cumplir que $-2 \leq c \leq 2$ tomamos $c \simeq 0.735$. \\

Por lo tanto el valor de $c$ para el cual $ P(X > c) = 0.1$ es $c \simeq 0.735$.

\section*{Ejercicio 2:}

Definimos la variable aleatoria:
\begin{center}
  $X$: tiempo de vida de un marcapasos.
\end{center}

Nuestra función $f$ de densidad de probabilidad tiene una distribución exponencial
con media de 16 años, por lo tanto, nuestro parámetro $\alpha$ viene dado por:
$\alpha = \frac{1}{16}$. Entonces nuestra función $f$ nos queda:

\[
  f(x) = \begin{cases}
    \frac{1}{16}e^{(-1/16)x} - x & \text{si } x > 0          \\
    0                            & \text{en caso contrario }
  \end{cases}
\]

Y nuestra Función de Distribución Acumulada es:

\[
  F(x) = \begin{cases}
    1 - e^{(-1/16)x} & \text{si } x > 0          \\
    0                            & \text{en caso contrario }
  \end{cases}
\]

1) ¿Cuál es la probabilidad de que a una persona a la que se le ha implantado el
marcapasos se le deba reimplantar otro antes de los 20 años? \\

Para esto debemos calcular
\begin{align*}
  P(X < 20) & = F(20)             \\
            & = 1 - e^{(-1/16)20} \\
            & \simeq 0.7134
\end{align*}

Por lo tanto, la probabilidad es de 0.7134. \\

2) Si el marcapasos lleva funcionando correctamente 5 años en un paciente, ¿cuál
es la probabilidad de que haya que cambiarlo antes de 25 años? \\

Queremos hallar $P(X \leq 25 | X \geq 5) = \frac{P(5 \leq X \leq 25)}{P(X \geq 5)}$:
\begin{align*}
  P(5 \leq X \leq 25) & = F(25) - F(5)                             \\
                      & = (1 - e^{(-1/16)25}) - (1 - e^{(-1/16)5}) \\
                      & \simeq 0.522
\end{align*}

\begin{align*}
  P(X \geq 5) &= 1 - P(X < 5) \\
              &= 1 - F(5) \\
              &= 1 - (1- e^{(-1/16)5}) \\
              &\simeq 0.7316
\end{align*}

Luego:

\begin{equation*}
  P(X \leq 25 | X \geq 5) = \frac{P(5 \leq X \leq 25)}{P(X \geq 5)} 
                          \simeq \frac{0.522}{0.7316} 
                          \simeq 0.7135
\end{equation*}

Por lo tanto, la probabilidad de que haya que reemplazarlo antes de los 25 años
es de 0.7135. 

\section*{Ejercicio 3:}
\textbf{\underline{Apartado a):}} \\

Definimos la variable aleatoria: \\

$X:$ duración en horas de un láser \\

$ X $ tiene una \textbf{distribución normal} con parámetros $ \mu = 7000 $ y 
$ \sigma = 600 $. Queremos hallar $ P(X < 5000) $, para ello debemos convertir 
esta variable $ X $ que tiene una distribución normal, en una nueva variable $ Z $
con distribución normal \textbf{estandarizada}:

\begin{align*}
  P(X < 5000) &= P(\frac{X - \mu}{\sigma} < \frac{5000 - \mu}{\sigma}) \\
              &= P(\frac{X - 7000}{600} < \frac{5000 - 7000}{600})     \\
              &= P(Z < -3.\wideparen{3}) \rightarrow (\text{llamamos a Z} = \frac{X - 7000}{600})
\end{align*}

Ahora, como la variable aleatoria $ Z $ tiene una distribución normal estandarizada,
observamos la tabla y hallamos que: $ P(Z < -3.\wideparen{3}) = 0.0004 $. \\

Por lo tanto, la probabilidad de que el láser falle antes de las 5000hs es de 0.0004. \\

\textbf{\underline{Apartado b):}} \\

En este caso, queremos hallar el $ k $ para el cual $ P(X > k) = 0.95 $.

\begin{equation*}
  P(X > k) = 0.95 \iff P(X < k) = 0.05
\end{equation*}

\begin{align*}
  P(X < k) &= P(\frac{X - \mu}{\sigma} < \frac{k - \mu}{\sigma}) \\
           &= P(\frac{X - 7000}{600} < \frac{k - 7000}{600})     \\
           &= P(Z < \frac{k - 7000}{600})     \\
           &= 0.05
\end{align*}

Observando la tabla, vemos que se debe tomar $ z = -1.65 $, y luego:

\begin{align*}
  -1.65           &= \frac{k - 7000}{600} \\
  600 \cdot -1.65 &= k - 7000 \\
  -990            &= k - 7000 \\
  6010            &= k 
\end{align*}

\textbf{\underline{Apartado c):}} \\

Definimos una variable aleatoria: \\

$ X': $ n° de láseres que siguen funcionando después de 7000 hs. \\

$X' $ tiene una \textbf{distribución binomial} de parámetros $ n = 3 $ y 
$ p = P(X \geq 7000) $. Ahora calculamos $ p $:

\begin{align*}
  P(X \geq 7000) &= 1 - P(X < 7000) \\
                 &= 1 - P(Z < \frac{7000 - 7000}{600}) \\
                 &= 1 - P(Z < 0) \\
                 &= 1 - 0.5 \rightarrow \text{observando la tabla} \\
                 &= 0.5
\end{align*}

Luego resulta, $ p = 0.5 $. Ahora queremos hallar: \\

$ P(X' = 3) = \binom{3}{3} \cdot p^3 \cdot (1 - p)^0 = (0.5)^3 = 0.125 $ \\

Por lo tanto, la probabilidad de que los 3 láseres sigan funcionando después de 
7000 hs es de 0.125. 

\newpage

\section*{Ejercicio 4:}

Sea $ X$: temperatura en celsius. \\

$ X $ tiene una distribución uniforme en el intervalo (15, 21), por lo tanto, su
su fdp viene dada por: 

\begin{equation*}
  f_X(x) = \frac{1}{6}
\end{equation*}

Luego tenemos la variable aleatoria: \\

$Y$: temperatura en farenheit. \\

Y también se tiene que $ H(X) = Y = \frac{9}{5}X + 32 $. Queremos hallar la fdp $f_Y$. \\

Observemos que por teorema tenemos que:

\begin{equation*}
  f_Y(x) = f_X(x) \cdot \Big|\frac{\partial x}{\partial y}\Big|
\end{equation*}

Donde $ x = H^{-1}(y) = \frac{5}{9}y - \frac{160}{9} $. Por lo tanto:

\begin{equation*}
  f_Y(x) = \frac{1}{6} \cdot \frac{5}{9} = \frac{5}{54}
\end{equation*}

Y observemos que, de hecho, $f_y$ es una fdp ya que:

\begin{itemize}
  \item $f_Y(x) \geq 0 \; \forall x \; | \; 59 \leq x \leq 69.8 $ (Obs. que si $15 \leq X \leq 21 $ entonces $59 \leq H(X) = Y \leq 69.8$)
  \item $ \int_{-\infty}^{+\infty} \frac{5}{54} dx =  
                  \int_{59}^{69.8} \frac{5}{54} dx = 
                  (69.8 - 59) \cdot \frac{5}{54}   = 1 $
\end{itemize}

\end{document}