\documentclass[11pt]{article}
\usepackage[a4paper, margin=2.54cm]{geometry}
\usepackage[utf8]{inputenc}
\usepackage[spanish, mexico]{babel}
\usepackage[spanish]{layout}
\usepackage[article]{ragged2e}
\usepackage{textcomp}
\usepackage{amsmath}
\usepackage{amsfonts}

\setlength{\parindent}{0pt}

\title{Trabajo Práctico: Unidad 4}
\author{Mellino, Natalia \and Farizano, Juan Ignacio}
\date{}



\begin{document}

\maketitle

\section*{Ejercicio 1}

\textbf{Apartado a)} \\

Sea $ X $ la variable aleatoria:

\begin{center}
  $ X $: "Número de interrupciones diarias"
\end{center}

Sabemos que $ F(4) = 0.97 $, entonces se tiene que:

% Alinear las ecuaciones después

\begin{align*}
  P(X = 0) + P(X = 1) + P(X = 2) + P(X = 3) + a & =  0.97 \\
  0.32 + 0.35 + 0.18 + 0.08 + a                 & =  0.97 \\
  a                                             & =  0.04
\end{align*}

Entonces podemos deducir que:

\begin{align*}
  b & =  1 - (P(X = 0) + P(X = 1) + P(X = 2) + P(X = 3) + P(X = 4) + P(X = 6)) \\
  b & =  1 - (0.32 + 0.35 + 0.18 + 0.08 + 0.04 + 0.01)                         \\
  b & =  0.02
\end{align*}

\textbf{Apartado b)} \\

\textbf{I)} Queremos hallar la probabilidad de que $ P(X \leq 4) $:

\[ P(X \leq 4) = \sum_{i=0}^{4}P(X = i) = 0.97 \]

Por lo tanto, la probabilidad de que haya a lo sumo 4 interrupciones es de 0.97.\\

\textbf{II)} Queremos hallar la probabilidad de que $ P(X \geq 5) $, esto es:

\begin{center}
  $ P(X = 5) + P(X = 6) = 0.02 + 0.01 = 0.03$
\end{center}

Por lo tanto, la probabilidad de que haya por lo menos 5 interrupciones es de 0.03. \\

\textbf{Apartado c)} \\

Primero debemos calcular la esperanza matemática $ (E(X)) $:

\[ E(X) = \sum_{i=0}^{6} i \cdot P(X = i) \]

$ E(X) = 0.35 + 2 \cdot 0.18 + 3 \cdot 0.08 + 4 \cdot 0.04 + 5 \cdot 0.02 + 6 \cdot 0.01 $ \\
$ E(X) = 1.27 $ \\

Ahora calculamos la varianza $ (V(X)) $ para luego poder calcular la desviación estándar:

\[  V(X) = E(X^{2}) - E(X)^{2} \]

$ E(X^{2}) = 1^2 \cdot 0.35 + 2^2 \cdot 0.18 + 3^2 \cdot 0.08 + 4^2 \cdot 0.04 + 5^2 \cdot 0.02 + 6^2 \cdot 0.01 $ \\
$ E(X)^2 = 3.29 $ \\

Luego $ V(X) = 3.29 - (1.27)^2 = 1.6771 $ \\

Entonces para calcular la desviación estándar $ (\sigma) $ simplemente hacemos:

\begin{center}
  $ \sigma = \sqrt{V(X)} = \sqrt{1.6771} \simeq 1.295 $
\end{center}

De los resultados obtenidos interpretamos que el número promedio de interrupciones
por día es de 3.29, con una desviación de aproximadamente 1.295. Este último resultado
nos brinda información acerca de qué tan alejados están los datos con respecto al
promedio. \\

\textbf{Apartado d)} \\

Los resultados hallados en c) representan \textbf{parámetros} porque son calculados con respecto
a la \textbf{población}, si fueran estadísticos estos hubieran sido
calculados a partir de una \textbf{muestra}.

\section*{Ejercicio 2}

Definimos el experimento $E:$ se realiza una transmisión. \\

\textbf{Probabilidad de cometer un error sin el sistema corrector-error:}

Sean los sucesos: \\
$A_i:$ Se transmite $i,\;i=0,1$ \\
$B_i:$ Se interpreta $i,\;i=0,1$ \\
Sabemos que $P(A_0) = p$ y $P(A_1) = (1-p)$

La probabilidad de interpretar un bit erróneamente es de $\epsilon$, por lo tanto:
\begin{equation*}
  P(B_1/A_0) = P(B_0/A_1) = \epsilon
\end{equation*}

Se quiere calcular la probabilidad de tomar una decisión errónea:
\begin{align*}
  &P(A_0 \cap B_1) = P(B_1/A_0)P(A_0) = \epsilon \cdot p \\
  &P(A_1 \cap B_0) = P(B_0/A_1)P(A_1) = \epsilon \cdot (1-p)
\end{align*}
Sumando las dos ecuaciones anteriores, se obtiene la probabilidad deseada.
\begin{equation*}
  P(A_0 \cap B_1) + P(A_1 \cap B_0) = \epsilon \cdot p + \epsilon \cdot (1-p) =
  \epsilon \cdot p + \epsilon - p \cdot \epsilon = \epsilon
\end{equation*}

\textbf{Probabilidad de cometer un error con el sistema corrector-error:}

Definimos el suceso $A$: "se interpreta un bit erróneamente". La probabilidad de 
dicho suceso es $ P(A) = \epsilon, \; P(\overline{A}) = 1 - \epsilon $, dada
por el apartado anterior. 

Sea la variable aleatoria, $X$:"número de veces que ocurrió el suceso $A$
en las $n$ repeticiones del experimento $E$".

Sabemos que un error se cometió en la decisión si 2 o más de los bits son erróneos.
Entonces, nuestra v.a. $X$ presenta una \textbf{distribución binomial} de parámetros
$ n = 3, \; p = \epsilon $. \\
$ X \sim B(3, \epsilon) $
\begin{align*}
  P(X \geq 2) = P(X = 2) + P(X = 3) & = \binom{3}{2} \cdot  \epsilon^2 \cdot(1 - \epsilon)^1 + \binom{3}{3} \cdot \epsilon^3 \cdot (1 - \epsilon)^0 \\
                                    & = 3 \cdot \epsilon^2 \cdot (1 - \epsilon) + \epsilon^3
\end{align*}

Por lo tanto, la probabilidad de cometer un error con el sistema corrector-error
es de \newline $ 3 \cdot \epsilon^2 \cdot (1 - \epsilon) + \epsilon^3 $. \\

\textbf{Conclusiones:} 

A costo de un mayor tiempo $T$ para transmitir la misma información, se puede 
obtener una menor probabilidad de error con el sistema corrector-error si y solo si:
\begin{align*}
  & 3\cdot\epsilon^2(1-\epsilon) + \epsilon^3 < \epsilon \iff 3\cdot\epsilon^2 -
  2\cdot\epsilon^3 + \epsilon^3 < \epsilon \iff 3\cdot\epsilon^2 - 
  2\cdot\epsilon^3 < \epsilon \iff \epsilon\cdot(3\cdot\epsilon-2\cdot\epsilon^2) <
  \epsilon \iff \\
  & 3\cdot\epsilon-2\cdot\epsilon^2 < 1 \iff \epsilon < \frac{1}{2} \;\lor\; \epsilon > 1
  \underset{\epsilon \leq 1}{\iff} \epsilon < \frac{1}{2}
\end{align*}
Por lo tanto, el sistema corrector-error resulta más eficiente si la probabilidad
de tomar una decisión errónea es menor a $\frac{1}{2}$.

\end{document}