\documentclass[11pt]{article}
\usepackage[a4paper, margin=2.54cm]{geometry}
\usepackage[utf8]{inputenc}
\usepackage[spanish, mexico]{babel}
\usepackage[spanish]{layout}
\usepackage[article]{ragged2e}
\usepackage{textcomp}
\usepackage{amsmath}

% ============================================================================
% ============================================================================
% ============================================================================

\title{
  Trabajo Práctico: Unidad 3
}
\author{
  Farizano, Juan Ignacio\\
  \and
  Mellino, Natalia
}
\date{}

% ============================================================================
% ============================================================================
% ============================================================================

\begin{document}

\maketitle
\noindent\rule{\textwidth}{1pt}

% ============================================================================
% ============================================================================
% ============================================================================

\begin{justify}
  \textbf{Ejercicio 1:} \\
  Definimos el experimento $\epsilon$: probar un programa \\
  Con los siguientes sucesos:
  \begin{itemize}
    \item $A_1$: Encontrar errores importantes al correr por primera vez el programa.
    \item $B_1$: Encontrar errores menores al  correr por primera vez el programa.
    \item $C_1$: No encontrar errores al correr por primera vez el programa.
 
    \item $A_2$: Encontrar errores importantes al correr el programa por segunda vez.
    \item $B_2$: Encontrar errores menores al correr el programa por segunda vez.
    \item $C_2$: No encontrar errores al correr el programa por segunda vez.
  \end{itemize}
    Dados los sucesos definidos y tomando los datos que se presentan en el 
    enunciado podemos decir que :
  \begin{equation*}
  P(A_1) = 0.6,\; P(B_1) = 0.3,\; P(C_1) = 0.1 \\
  \end{equation*}
\end{justify}

% ============================================================================

\begin{justify}
  \textbf{Apartado a):} \\
  Para hallar las intersecciones de los sucesos pedidas, realizaremos los 
  siguientes cálculos, con la información provista de la tabla presente en el
  enunciado y utilizando la fórmula de probabilidad condicional.
\end{justify}

\begin{align*}
    &P(A_1 \cap A_2) = P(A_2/A_1) * P(A_1) = 0.3 * 0.6 = 0.18 \\
    &P(A_1 \cap B_2) = P(B_2/A_1) * P(A_1) = 0.5 * 0.6 = 0.3 \\
    &P(A_1 \cap C_2) = P(C_2/A_1) * P(A_1) = 0.2 * 0.6 = 0.12 \\
    &P(B_1 \cap A_2) = P(A_2/B_1) * P(B_1) = 0.1 * 0.3 = 0.03 \\
    &P(B_1 \cap B_2) = P(B_2/B_1) * P(B_1) = 0.3 * 0.3 = 0.09 \\
    &P(B_1 \cap C_2) = P(C_2/B_1) * P(B_1) = 0.6 * 0.3 = 0.18 \\
    &P(C_1 \cap A_2) = P(A_2/C_1) * P(C_1) = 0 * 0.1 = 0 \\
    &P(C_1 \cap B_2) = P(B_2/C_1) * P(C_1) = 0.2 * 0.1 = 0.02 \\
    &P(C_1 \cap C_2) = P(C_2/C_1) * P(C_1) = 0.8 * 0.1 = 0.08 \\
\end{align*}

\begin{justify}
  Entonces la tabla nos queda de la siguiente manera:
\end{justify}

\begin{center}
  \begin{tabular}{| c | r | r | r |}
    \hline
    & $A_2$ & $B_2$ & $C_2$ \\ \hline
    $A_1$ & 0.18 & 0.30 & 0.12 \\ \hline
    $B_1$ & 0.03 & 0.09 & 0.18 \\ \hline
    $C_1$ & 0.00 & 0.02 & 0.08 \\ \hline
  \end{tabular}
\end{center}

% ============================================================================

\begin{justify}
  \textbf{Apartado b):} \\
  Observemos que los sucesos $A_1$, $B_1$ y $C_1$ son mutuamente excluyentes
  entre sí y además conforman una partición del espacio muestral A ya que:
  \begin{itemize}
    \item Todas sus intersecciones son vacías dos a dos
    ($A_1 \cap B_1 = \emptyset,\;A_1 \cap C_1 = \emptyset,\;B_1 \cap C_1 = \emptyset$)
    \item $P(A_1) + P(B_1) + P(C_1) = 0.6 + 0.3 + 0.1 = 1$
    \item Las probabilidades de todos los sucesos son mayores a 0.
  \end{itemize}
  Luego podemos describir al suceso $A_2$ como la unión de sucesos mutuamente
  excluyentes:
\end{justify}
\begin{equation*}
  A_2 = (A_2 \cap A1) \cup (A_2 \cap B_1) \cup (A_2 \cap C_1)
\end{equation*}
\begin{justify}
  Y por lo tanto:
\end{justify}
\begin{align*}
  P(A_2) &= P(A_2 \cap A_1) + P(A_2 \cap B_1) + P(A_2 \cap C_1) 
  = 0.18 + 0.03 + 0 = 0.21
\end{align*}

% ============================================================================

\begin{justify}
  \textbf{Apartado c):} \\
  Utilizando la fórmula de probabilidad condicional basta plantear:
\end{justify}
\begin{equation*}
  P(B_1/A_2) = P(B_1 \cap A_2) / P(A_2) = 0.03 / 0.21 = 1/7
\end{equation*}
\begin{justify}
  Por lo tanto, la probabilidad de encontrar un error menor en la primera
  prueba sabiendo que el de la segunda fue importante es de 1/7.
\end{justify}

% ============================================================================

\begin{justify}
  \textbf{Apartado d):} \\
  Sabemos que dos sucesos son independientes sí y sólo sí:
\end{justify}
\begin{equation*}
  P(A \cap B) = P(A) * P(B)
\end{equation*}
\begin{justify}
  Por lo tanto basta plantear las siguientes igualdades:
\end{justify}
\begin{align*}
  &P(A_1 \cap A_2) = 0.18 \\
  &P(A_1) * P(A_2) = 0.6 * 0.21 = 0.126
\end{align*}
\begin{justify}
  $0.18 \neq 0.126$ por lo tanto $A_1$ y $A_2$ no son independientes.
\end{justify}

\begin{equation*}
  P(A_1 \cap B_2) = 0.03
\end{equation*}
\begin{justify}
  Necesitamos hallar $P(B_2)$, procedemos de la misma forma que en el apartado b)
\end{justify}
\begin{align*}
  &P(B_2) = P(B2 \cap A1) + P(B_2 \cap B_1) + P(B_2 \cap C_1) = 0.03 + 0.09 + 0.18 = 0.41 \\
  &P(A_1) * P(B_2) = 0.6 * 0.41 = 0.246
\end{align*}
\begin{justify}
  $0.03 \neq 0.246$ por lo tanto $A_1$ y $B_2$ no son independientes.
\end{justify}

\begin{align*}
  &P(A_1 \cap C_2) = 0.12\\
  &P(C_2) = P(C_2 \cap A_1) + P(C_2 \cap B_1) + P(C_2 \cap C_1) = 0.12 + 0.18 + 0.08 = 0.38 \\
  &P(A_1) * P(C_2) = 0.6 * 0.38 = 0.228
\end{align*}
\begin{justify}
  $0.12 \neq 0.228$ por lo tanto $A_1$ y $C_2$ no son independientes.
\end{justify}

\begin{align*}
  &P(B_1 \cap A_2) = 0.03 \\
  &P(B_1) * P(A_2) = 0.3 * 0.21 = 0.063
\end{align*}
\begin{justify}
  $0.03 \neq 0.063$ por lo tanto $B_1$ y $A_2$ no son independientes.
\end{justify}

\begin{align*}
  &P(B_1 \cap B2) = 0.09 \\
  &P(B_1) * P(B_2) = 0.3 * 0.41 = 0.123
\end{align*}
\begin{justify}
  $0.09 \neq 0.123$ por lo tanto $B_1$ y $B_2$ no son independientes.
\end{justify}

\begin{align*}
  &P(B1 \cap C2) = 0.18 \\
  &P(B_1) * P(C_2) = 0.3 * 0.38 = 0.114
\end{align*}
\begin{justify}
  $0.18 \neq 0.114$ por lo tanto $B_1$ y $C_2$ no son independientes.
\end{justify}

\begin{align*}
  &P(C_1 \cap A_2) = 0 \\
  &P(C_1) * P(A_2) = 0.1 * 0.21 = 0.021
\end{align*}
\begin{justify}
  $0 \neq 0.021$ por lo tanto $C_1$ y $A_2$ no son independientes.
\end{justify}

\begin{align*}
  &P(C_1 \cap B_2) = 0.02 \\
  &P(C_1) * P(B_2) = 0.1 * 0.41 = 0.041
\end{align*}
\begin{justify}
  $0.02 \neq 0.041$ por lo tanto $C_1$ y $B_2$ no son independientes.
\end{justify}

\begin{align*}
  &P(C_1 \cap C_2) = 0.08 \\
  &P(C_1) * P(C_2) = 0.1 * 0.38 = 0.038
\end{align*}
\begin{justify}
  $0.08 \neq 0.038$ por lo tanto $C_1$ y $C_2$ no son independientes.
\end{justify}

\begin{justify}
  En conclusión: los resultados de la primera prueba, no son independientes con
    los resultados de la segunda.
\end{justify}

% ============================================================================
% ============================================================================
% ============================================================================

\begin{justify}
  \textbf{Ejercicio 2:} \\
  Definimos el experimento $\epsilon:$ se realiza una transmisión. \\
  Con los siguientes sucesos: \\
  $A_i:$ Recibir $i,\;i=0,1$ \\
  $B_i:$ Enviar $i,\;i=0,1$ \\
  Sabemos que tenemos la misma probabilidad de enviar un 0 que de enviar un 1, 
  observemos que los sucesos $B_0$ y $B_1$ conforman una partición del espacio
  muestral asociado a $\epsilon$. Por lo tanto: $P(B_0) + P(B_1) = 1$
  y como ambos sucesos tienen la misma probabilidad se deduce que $P(B_i) = 0.5$. 
  $i,\;i=0,1$\\
  Ahora, queremos hallar la probabilidad de cometer un error en la transmisión.
  Por lo tanto este suceso lo podemos describir como:

  $(A_0 \cap B1)\,\cup\,(A_1 \cap B_0) \rightarrow$ Observemos que estos sucesos
  $((A_0 \cap B_1) \;\text{y}\; (A_1 \cap B_0))$ son mutuamente excluyentes
  y por lo tanto su intersección es vacía. Entonces la probabilidad la podemos
  expresar como:
\end{justify}
\begin{equation*}
  P(A_0 \cap B_1) + P(A_1 \cap B_0) = P(A_0/B_1)P(B_1) + P(A_1/B_0)P(B_0)
\end{equation*}
\begin{justify}
  Ahora supongamos que las probabilidades $P(A_0/B_1)$ y $P(A_1/B_0$) son
  $\epsilon_1$ y $\epsilon_2$ respectivamente, para algún $0 \leq \epsilon_i
  \leq 1, i = 1,2$.\\
  Entonces la probabilidad de cometer un error en la transmisión es:
\end{justify}
\begin{equation*}
  \epsilon_1 * 0.5 + \epsilon_2 * 0.5 = 0.5 * (\epsilon_1 + \epsilon_2)
\end{equation*}

% ============================================================================
% ============================================================================
% ============================================================================

\begin{justify}
  \textbf{Ejercicio 3:} \\
  Definimos el experimento $\epsilon:$ se realiza una transmisión. \\
  Con los siguientes sucesos: \\
  $A_i:$ La entrada fue $i,\;i=0,1$ \\
  $B_i:$ La salida fue $i,\;i=0,1$ \\
  Sabemos que $P(A_0) = (1-p)$ y $P(A_1) = p$
\end{justify}

\begin{justify}
  \textbf{Apartado a):} \\
  Dado que la probabilidad de tomar decisiones erróneas es $\epsilon$,
  podemos deducir que la probabilidad de tomar una desición correcta es
  $1 - \epsilon$. Entonces utilizando la fórmula de probabilidad condicional
  nos queda:
  \begin{align*}
    &P(A_0 \cap B_0) = P(B_0/A_0)P(A_0) = (1 - \epsilon) * (1-p) \\
    &P(A_0 \cap B_1) = P(B_1/A_0)P(A_0) = \epsilon * (1-p) \\
    &P(A_1 \cap B_0) = P(B_0/A_1)P(A_1) = \epsilon * p \\
    &P(A_1 \cap B_1) = P(B_1/A_1)P(A_1) = (1 - \epsilon) * p
  \end{align*}
\end{justify}

% ============================================================================

\begin{justify}
  \textbf{Apartado b):} \\
  De las igualdades obtenidas en el apartado anterior podemos deducir lo siguiente:
  \begin{align*}
    &P(A_0 \cap B_0) = P(B_0)P(A_0) = (1 - \epsilon) * (1-p) \\
    &P(A_1 \cap B_1) = P(B_1)P(A_1) = (1 - \epsilon) * p \\
    &P(A_0 \cap B_1) = P(B_1)P(A_0) = \epsilon * (1-p) 
    \Rightarrow P(B_1)*(1-p) = \epsilon*(1-p) \Rightarrow P(B_1) = \epsilon \\
    &P(A_1 \cap B_0) = P(B_0)P(A_1) = \epsilon * p
    \Rightarrow P(B_O)*p = \epsilon*p \Rightarrow P(B_0) = \epsilon
  \end{align*}
  Luego:
  \begin{equation*}
    P(B_0) + P(B_1) = \epsilon + \epsilon = 2*\epsilon
  \end{equation*}
  Como $B_0$ y $B_1$ conforman una partición del espacio muestral asociado al
  experimento deben ser mutuamente excluyentes
  y $P(B_0) + P(B_1) = 1 = 2*\epsilon \Rightarrow \epsilon = \frac{1}{2}$
\end{justify}

% ============================================================================

\begin{justify}
  \textbf{Apartado c):} \\
  Los sucesos que resultan ser mutuamente excluyentes son los sucesos $A_0$ y
  $A_1$, y $B_0$ y $B_1$, pues observemos que es imposible que ambos puedan
  suceder al mismo tiempo.
\end{justify}

% ============================================================================

\begin{justify}
  \textbf{Apartado d):} \\
  Utilizando el Teorema de Bayes calculamos las probabilidades condicionales:
  \begin{equation*}
    P(A_0/B_1) = \frac{P(B_1 \cap A_0)}{P(B_1/A_0)P(A_0) + P(B_1/A_1)P(A_1)}
    = \frac{\epsilon * (1-p)}{\epsilon * (1-p) + (1-e) * p} =
    \frac{\epsilon - p * \epsilon}{\epsilon + p - 2 * \epsilon * p}
  \end{equation*}
  \begin{equation*}
    P(A_1/B_1) = \frac{P(B_1 \cap A_1)}{P(B_1/A_0)P(A_0) + P(B_1/A_1)P(A_1)}
    = \frac{(1 - \epsilon) * p}{\epsilon * (1-p) + (1-e) * p} =
    \frac{p - p * \epsilon}{\epsilon + p - 2 * \epsilon * p}
  \end{equation*}
  Luego, observemos que: 
  \begin{equation*}
    P(A_0/B_1) < P(A_1/B_1) \iff 
    \frac{\epsilon - p * \epsilon}{\epsilon + p - 2 * \epsilon * p} <
    \frac{p - p * \epsilon}{\epsilon + p - 2 * \epsilon * p} \iff 
    \epsilon - p * \epsilon < p - p * \epsilon \iff \epsilon < p
  \end{equation*}
  Entonces si $\epsilon < p$ la entrada más probable va a ser 1.
  Análogamente, si $\epsilon > p$ la entrada más probable va a ser 0.
\end{justify}

\end{document}