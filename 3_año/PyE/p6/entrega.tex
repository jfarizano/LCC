\documentclass[11pt]{article}
\usepackage[a4paper, margin=2.54cm]{geometry}
\usepackage[utf8]{inputenc}
\usepackage[spanish, mexico]{babel}
\usepackage[spanish]{layout}
\usepackage[article]{ragged2e}
\usepackage{textcomp}
\usepackage{amsmath}
\usepackage{amsfonts}
\usepackage{yhmath}

\setlength{\parindent}{0pt}

\title{Trabajo Práctico: Unidad 6}
\author{Mellino, Natalia \and Farizano, Juan Ignacio}
\date{}



\begin{document}
\maketitle

%==============================================================================

\section*{Ejercicio 1}

Definimos la variable aleatoria: 

\[
  X_i = \begin{cases}
            1 & \text{si gano en la tirada i} \\
            0 & \text{en caso contrario}
        \end{cases}  
\]

Para todo $i = 1,...50$. La probabilidad de que nuestro número elegido
salga en la tirada $i$ es $p = \frac{1}{10}$. Además, observemos que
para $X_i$ sólo hay dos resultados posibles: $1$ ó $0$ (pues sólo hay
dos opciones: o sale nuestro numero ó no sale). Por lo tanto, cada v.a.
$X_i$ es una variable aleatoria con distribución binomial: $X_i \sim B(n = 1, p = \frac{1}{10})$.
Además, es claro que todas las $X_i$ son independientes entre sí, ninguna depende
de la otra. \\

El ejercicio nos pide hallar la probabilidad de no haber perdido dinero
después de 50 tiradas, esto sólo sucede si ganamos al menos 9 tiradas (si ganamos 8 o menos el saldo es negativo), 
es decir, queremos hallar $P(X_1 + ... + X_{50} \geq 9)$. Ahora, 
tenemos todas las hipótesis para poder usar el \textbf{Teorema Central del Límite}. Sea $Y$
la nueva variable aleatoria:

\begin{equation*}    
    Y = X_1 + ... + X_{50}
\end{equation*}

Sean $\mu_i = E(X_i) = p = \frac{1}{10}$,  y $\sigma_i^2 = V(X_i) = p(1 - p ) = \frac{9}{100}$, usando
el Teorema se tiene que $Y$ tiene una distribución aproximadamente normal con parámetros:

\begin{equation*}
    \mu = \displaystyle\sum_{i=1}^{50} \mu_i = 50 \cdot \frac{1}{10} = 5
\end{equation*}

\begin{equation*}
    \sigma = \sqrt{\displaystyle\sum_{i=1}^{50} \sigma_i^2} = \sqrt{50 \cdot \frac{9}{100}} \simeq 2.1213
\end{equation*}

Luego:

\begin{align*}
    P(Y \geq 9) & = 1 - P(Y < 9) \\
                & = 1 - P(Z < \frac{ 9 -\mu}{\sigma}) \rightarrow \text{estandarizamos} \\
                & = 1 - P(Z < 1.88)  \\
                & = 1 - 0.9699 \rightarrow \text{usando la tabla} \\
                & = 0.03
\end{align*}

Por lo tanto, la probabilidad de no perder dinero después de 50 tiradas, es de
aproximadamente 0.03.

%==============================================================================

\newpage

\section*{Ejercicio 2}

\subsection*{Apartado a):}

Definimos las siguientes variables aleatorias:

\[
    X_i = \begin{cases}
            1 & \text{si el pasajero $i$ decidió viajar} \\
            0 & \text{en caso contrario}
          \end{cases}
\]

Con $i = 1,..., 400$. Ahora, la probabilidad de que un pasajero decida
viajar es $p = 0.5$, luego, $X_i \sim B(n = 1, p = 0.5)$. Asumiendo que las
$X_i$ son independientes entre sí (que ningún pasajero dependa de otro para viajar o no), 
podemos usar el \textbf{Teorema Central del Límite} para calcular:

\begin{equation*}
    P(X_1 + ... + X_{400} > 230)
\end{equation*}

Sean $\mu_i = E(X_i) = p = 0.5$ y $\sigma_i^2 = V(X_i) = p(1 - p) = 0.25$, 
tenemos que $X_1 + ... + X_{400}$ tiene una distribución aproximadamente normal
con parámetros:

\begin{equation*}
    \mu = \displaystyle\sum_{i=1}^{400} \mu_i = 400 \cdot 0.5 = 200
\end{equation*}

\begin{equation*}
    \sigma = \sqrt{\displaystyle\sum_{i=1}^{400} \sigma_i^2} = \sqrt{400 \cdot 0.25} = 10
\end{equation*}

Luego definiendo $Y = X_1 + ... + X_{400}$, se tiene que:

\begin{align*}
    P(Y > 230) & = 1 - P(Y \leq 230) \\
               & = 1 - P(Z \leq \frac{230 - 200}{10}) \rightarrow \text{estandarizamos} \\
               & = 1 - P(Z \leq 3) \\
               & = 1 - 0.9987 \rightarrow \text{usando la tabla} \\
               & = \frac{13}{10000}
\end{align*}

Por lo tanto, la probabilidad de obtener un overbooking es de $\frac{13}{10000}$.

\subsection*{Apartado b)}

Definimos una nueva variable aleatoria, $X'$: número de compañías que tuvieron overbooking. \\

Tenemos 10 compañías, todas en las mismas condiciones, entonces la probabilidad de que una
tenga un overbooking, es la probabilidad hallada en el apartado anterior, $p = \frac{13}{10000}$.
Luego, es fácil ver que $X' \sim B(10, \frac{13}{10000})$. Nosotros quremos hallar $P(X' > 2)$, entonces:

\begin{align*}
    P(X' > 2) & = 1 - P(X' < 2) \\
              & = 1 - (P(X' = 0) + P(X' = 1)) \\
              & = 1 - \left(\left(\binom{10}{0} \cdot p^0 \cdot (1-p)^{10}\right) + \left(\binom{10}{1} \cdot p^1 \cdot (1-p)^9\right)\right) \\
              & = 1 - (0.987 + 0.0128) \\
              & = 0.0002
\end{align*}

Por lo tanto, la probabilidad de que al menos 2 compañías tengan overbooking es de 
0.0002.

\section*{Ejercicio 3}

\subsection*{Apartado a):}

De los datos proporcionados, podemos deducir las siguientes cosas: 

\begin{itemize}

\item El $20\%$ de las piezas tienen defectos tipo D2, esto quiere decir que
      $p_y(2) = 0.2$. 

\item El $15\%$ de las piezas tienen 1 defecto, tipo D1 y ninguno tipo D2, esto es:
      $P(1, 0) = 0.15$. 

\item $ E(X) = 0 \cdot p_x(0) + 1 \cdot p_x(1) = 0.3 \Rightarrow p_x(1) = 0.3 \Rightarrow p_x(0) = 1 - p_x(1) = 0.7$ 

\item El $50\%$ de las piezas que no tienen defectos tipo D1, tienen un defecto
      tipo D2, esto nos da una probabilidad condicional: 
      $P(Y = 1 / X = 0) = \frac{p(0, 1)}{p_x(0)} = 0.5 \Rightarrow \frac{p(0, 1)}{0.7} = 0.5 \Rightarrow p(0, 1)= 0.35 $ 

\item $ E(Y) = 0 \cdot p_y(0) + 1 \cdot p_y(1) + 2 \cdot p_y(2) = 0.8 \Rightarrow p_y(1) + 0.4 = 0.8 \Rightarrow p_y(1) = 0.4 $  

\item Luego como debe ser $p_y(0) + p_y(1) + p_y(2) = 1$, se deduce que $p_y(0) = 0.4$ 

\end{itemize}

Con estos datos tenemos lo suficiente para completar la tabla:

\begin{center}
    \begin{tabular}{| c | r | r | r | r |}
    \hline
    \textbf{X / Y} & \textbf{0} & \textbf{1} & \textbf{2} & $\mathbf{p_x(x)}$ \\ \hline
    \textbf{0} & 0.25 & 0.35 & 0.10 & 0.70 \\ \hline
    \textbf{1} & 0.15 & 0.05 & 0.10 & 0.30 \\ \hline
    $\mathbf{p_y(y)}$ & 0.40 & 0.40 & 0.20 & 1.00 \\ \hline
    \end{tabular}
\end{center}

\subsection*{Apartado b):}

$X$ e $Y$ no son independientes, observemos que:

\begin{itemize}
    \item $ P(0, 0) = 0.25$
    \item $p_x(0) \cdot p_y(0) = 0.4 \cdot 0.7 = 0.28$
\end{itemize}

Es decir, no se cumple que $P(x, y) = p_x(x) \cdot p_y(y)$, por lo tanto, 
las variables no son independientes. 
Para calcular el coeficiente correlación primero necesitamos el valor de $E(XY)$
\begin{align*}
    E(XY) & = \displaystyle\sum_{x = 0}^{1} \displaystyle\sum_{y = 0}^{2} x \cdot y \cdot p(x, y) \\
          & = 0 \cdot 0 \cdot 0.25 + 0 \cdot 1 \cdot 0.35 + 0 \cdot 2 \cdot 0.10
              + 1 \cdot 0 \cdot 0.15 + 1 \cdot 1 \cdot 0.05 + 1 \cdot 2 \cdot 0.10 \\
          & = 1 \cdot 1 \cdot 0.05 + 1 \cdot 2 \cdot 0.10 \\
          & = 0.25
\end{align*}

Una vez obtenido este valor, calculamos el coeficiente de correlación:
\begin{equation*}
    \rho = \frac{E(XY) - E(X)E(Y)}{\sqrt{V(X)V(Y)}} 
         = \frac{0.25 - (0.3 \cdot 0.8)}{\sqrt{0.21 \cdot 0.56}} = \frac{0.01}{0.342} = 0.029
\end{equation*}


\subsection*{Apartado c):}

Usamos la fórmula de probabilidad condicional:

\begin{align*}
    P(Y = 2 / X = 0) & = \frac{p(0, 2)}{p_x(x)} \\
                     & = \frac{0.1}{0.4} \\
                     & = 0.25
\end{align*}

Esto se interpreta como la probabilidad de que una pieza tenga 2 defectos
del tipo D2 dado que no tiene ningún defecto del tipo D1.

\subsection*{Apartado d):}
Definimos la variable aleatoria $Z$: ``costo de reparación por pieza", 
observemos que el costo de reparación se calcula: $Z = 3X + 4Y$, donde
$X$, e $Y$ son las variables aleatorias que representan la cantidad de 
defectos del tipo D1 y D2 respectivamente. \\

Para el costo esperado se tiene lo siguiente:

\begin{equation*}
    E(Z) = 3 \cdot E(X) + 4 \cdot E(Y) = 3 \cdot 0.3 + 4 \cdot 0.8 = 4.1
\end{equation*}

\begin{align*}
    V(Z) & = 9 \cdot V(X) + 16 \cdot V(Y) + 2 \cdot 3 \cdot 4 \cdot Cov(X, Y) \\
         & = 9 \cdot 0.21 + 16 \cdot 0.56 + 24 \cdot 0.01 \\
         & = 11.09 
\end{align*}

\end{document}